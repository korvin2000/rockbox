\subsection{Shortcuts}
\label{ref:Shortcutsplugin}

The Shortcuts Plugin allows you to jump to places within the file browser
without having to navigate there manually. The plugin works with
\fname{.link} files. A \fname{.link} file is just a text file with every
line containing the name of the file or the directory you want to quickly
jump to. All names should be full absolute names, i.e. they should start
with a \fname{/}. Directory names should also end with a \fname{/}.

\note{This plugin cannot read Microsoft Windows shortcuts (\fname{.lnk}
files). These are handled by a separate plugin; see
\reference{ref:Winshortcutsplugin}.}

\subsubsection{How to create \fname{.link} files}

You can use your favourite text editor to create a \fname{.link} file on the
PC an then copy the file to the \dap{}. Or you can use the context menu on
either a file or a directory in the file browser tree, and use the ``Add to
shortcuts'' menu option. This will append a line with the full name of the
file or the directory to the \fname{shortcuts.link} file in the root
directory of the \dap{}. (The file will be created if it does not exist
yet.) You can later rename the automatically created \fname{shortcuts.link}
file or move it to another directory if you wish. Subsequent calls of the
context menu will create it again.


\subsubsection{How to use \fname{.link} files, i.e. jump to desired places}

To use a \fname{.link} file just ``play'' it from the file browser. This will
show you a list with the entries in the file. Selecting one of them will
then exit the plugin and leave you within the directory selected, or with
the file selected in the file browser. You can then play the file or do
with it whatever you want. The file will not be ``played'' automatically.

If the \fname{.link} file contains only one entry no list will be shown, you
will directly jump to that location. The file \fname{shortcuts.link} in the
root directory is an exception. After ``playing'' it, the list will be shown
even if the file contains just one entry.

If the list you are seeing is from \fname{shortcuts.link} in the root
directory, you can delete the selected entry by pressing \ActionStdMenu.
Deleting entries from other \fname{.link} files is not possible.


\subsubsection{Advanced Usage}

Placing the line ``\#Display last path segments=n'' (where n is a number) in
the beginning of a \fname{.link} file will leave just the last n segments of
the entries when they are shown. For example, if n is chosen to be 1, then
the entry \fname{/MyMusic/collection/song.mp3} will be shown as
\fname{song.mp3}. This allows you to hide common path prefixes.

You can also provide a custom display name for each entry individually. To
do so, append a tabulator character after the entry's path followed by your
custom name. That name will then be used for showing the entry. For example:
\begin{example}
    /MyMusic/collection/song.mp3<TAB>My favourite song\symbol{33}
\end{example}
