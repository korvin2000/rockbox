% $Id$ %
\chapter{\label{ref:rockbox_interface}Quick Start}
\section{Basic Overview}
\subsection{The \daps{} controls}

% include the front image. Using \specimg makes this fairly easy,
% but requires to use the exact value of \specimg in the filename!
% The extension is selected in the preamble, so no further \Ifpdfoutput
% is necessary.
%
% The check looks for a png file -- we use png for the HTML manual, so that
% format needs to be present. It can also be used for the pdf manual, but
% usually we provide a pdf version of the file for that. Picking the correct
% one is done by LaTeX automatically, but for checking the filename we need to
% specify the extension.
\begin{center}
\IfFileExists{rockbox_interface/images/\specimg-front.png}
   {\Ifpdfoutput{\includegraphics[height=8cm,width=10cm,keepaspectratio=true]%
    {rockbox_interface/images/\specimg-front}}
   {\includegraphics{rockbox_interface/images/\specimg-front}}
   }
   {\color{red}{\textbf{WARNING!} Image not found}%
    \typeout{Warning: missing front image}
   }
\end{center}
\opt{HAVEREMOTEKEYMAP}{
  % spacing between the two pictures, could possibly be improved
  \begin{center}
  \IfFileExists{rockbox_interface/images/\specimg-remote.png}
    {\Ifpdfoutput{\includegraphics[height=5.6cm,width=10cm,keepaspectratio=true]{rockbox_interface/images/\specimg-remote}}
    {\includegraphics{rockbox_interface/images/\specimg-remote}}
    }
    {\color{red}{\textbf{WARNING!} Image not found}%
     \typeout{Warning: missing remote image}
    }
  \end{center}
}

Throughout this manual, the buttons on the \dap{} are labelled according to the
picture above.
\opt{touchscreen}{
The areas of the touchscreen in the 3$\times$3 grid mode are in turn referred as follows:
\begin{table}
    \centering
    \begin{tabular}{|c|c|c|}
	\hline
        \TouchTopLeft & \TouchTopMiddle & \TouchTopRight \\ [5ex]
	\hline
	\TouchMidLeft & \TouchCenter & \TouchMidRight \\ [5ex]
	\hline
	\TouchBottomLeft & \TouchBottomMiddle & \TouchBottomRight \\ [5ex]
	\hline
    \end{tabular}
\end{table}
}%
Whenever a button name is prefixed by ``Long'', a long press of approximately
one second should be performed on that button. The buttons are described in
detail in the following paragraph.
\blind{%
  Additional information for blind users is available on the Rockbox website at
  \wikilink{BlindFAQ}.

  %
  \opt{iriverh100}{
  Hold or lay the \dap{} so that the side with the joystick and LCD is facing
  towards you, and the curved side is at the top. The joystick functions as
  the \ButtonUp{}, \ButtonRight{}, \ButtonLeft{}, and \ButtonDown{} buttons when
  pressed in the appropriate direction. Pressing the joystick down functions as
  \ButtonSelect{}.
  On the right side of the \dap{} are the \ButtonOn{}, \ButtonOff{},
  \ButtonMode{} buttons, and the \ButtonHold{} switch. When this switch is
  switched towards the bottom of the \dap{}, hold is on, and none of the other
  buttons have any effect.

  On the left side is the \ButtonRec{} button. Above that is the internal microphone.

  On the top panel of the \dap{}, from left to right, you can find the
  following: headphone mini jack plug, remote port, Optical line-in, Optical line-out.

  On the bottom panel of the \dap{}, from left to right, you can find the
  following: power jack, reset switch, and USB port. In the event that your
  \dap{} hard locks, you can reset it by inserting a paper clip into the hole
  where the reset switch is.}
  %
  \opt{iriverh300}{
  Hold or lay the \dap{} so that the side with the button pad and
  LCD is facing towards you.  The buttons on the button pad are as follows:  top
  left corner: \ButtonOn{}, bottom left corner: \ButtonOff{}, top right corner:
  \ButtonRec, bottom right corner: \ButtonMode{}.  In the center of the button pad
  is a button labelled \ButtonSelect{}.  Surrounding the \ButtonSelect{} button are
  the \ButtonUp{}, \ButtonDown{}, \ButtonLeft{}, and \ButtonRight{} buttons.

  On the top panel of the \dap{}, from left to right, you can find the
  following: headphone mini jack plug, remote port, line-in, line-out.

  On the left hand side of the \dap{} is the internal microphone. Just underneath
  this is a small hole, the reset switch. In the event that your \dap{} hard locks,
  you can reset it by inserting a paper clip into the hole where the reset switch
  is.

  On the right hand side of the \dap{} is the \ButtonHold{} switch. When this is
  switched towards the bottom of the \dap{}, hold is on, and none of the other
  buttons have any effect.

  On the bottom panel of the \dap{}, from left to right, you can find the
  following:  power jack and two USB ports.  The USB port on the right is used
  to connect your \dap{} to your computer.  The USB port on the left is not
  used in Rockbox.
  }
  %
  \opt{mpiohd200}{
  Hold or lay the \dap{} so that the side with the LCD is facing towards you.
  On the right hand side there is a rocker switch at the top which serves as
  \ButtonRew{} and \ButtonFF{} when rocked up or down, respectively.
  Pressing the rocker in functions as the \ButtonFunc{} button. Below the rocker
  there are the \ButtonRec{} and \ButtonPlay{} buttons. At the bottom of the
  right panel there is the \ButtonHold{} switch. When this is switched towards the
  bottom of the \dap{}. hold is on, and none of the other buttons have any effect.

  On the top panel of the \dap{} there is another rocker which serves as the
  \ButtonVolDown{} and \ButtonVolUp{} buttons when pressed to the left or right,
  respectively.

  On the left hand side of the \dap{} there is a headphone mini jack plug at the top
  and a small hole at the bottom, the reset switch. In the event that your \dap{}
  hard locks, you can reset it by inserting a paper clip into the hole where the
  reset switch is.

  On the bottom panel of the \dap{}, from left to right, you can find the
  following: power jack, line-in jack and USB port (under rubber cover).
  }
  %
  \opt{ipod4g,ipodcolor,ipodvideo,ipodmini}{
  The main controls on the \dap{} are a slightly indented scroll wheel
  with a flat round button in the center. Hold the \dap{} with these controls
  facing you.

  The top of the player will have the following, from left to
  right:
  \opt{ipod4g,ipodcolor}{remote connector, headphone socket, \ButtonHold{}
    switch.}
  \opt{ipodvideo}{\ButtonHold{} switch, headphone socket.}
  \opt{ipodmini}{\ButtonHold{} switch, remote connector, headphone socket.}

  The dock connector that is used to connect your \dap{} to your computer is on
  the bottom panel of the \dap{}.

  The button in the middle of the wheel is called \ButtonSelect{}. You can
  operate the wheel by pressing the top, bottom, left or right sections,
  or by sliding your finger around it.  The top is \ButtonMenu{}, the bottom is
  \ButtonPlay{}, the left is \ButtonLeft{}, and the right is \ButtonRight{}.
  When the manual says to \ButtonScrollFwd{}, it means to slide your finger
  clockwise around the wheel. \ButtonScrollBack{} means to slide your finger
  counterclockwise. Note that the wheel is sensitive, so you will need to move
  slowly at first and get a feel for how it works.

  Note that when the \ButtonHold{} switch is pushed toward the center of the \dap{},
  hold is on, and none of the other controls do anything.  Be sure
  \ButtonHold{} is off before trying to use your player.
  }
  %
  \opt{ipod3g}{
  The main controls on the \dap{} are a slightly indented touch wheel
  with a flat round button in the center, and four buttons in a row above the
  touch wheel. Hold the \dap{} with these controls
  facing you.

  The top of the player will have the following, from left to
  right: remote connector, headphone socket, \ButtonHold{} switch.

  The dock connector that is used to connect your \dap{} to your computer is on
  the bottom panel of the \dap{}.

  The button in the middle of the wheel is called \ButtonSelect{}. You can
  operate the wheel by sliding your finger around it.  The row of
  buttons consists of, from left to right, the \ButtonLeft{},
  \ButtonMenu{}, \ButtonPlay{}, and \ButtonRight{} buttons.
  When the manual says to \ButtonScrollFwd{}, it means to slide your finger
  clockwise around the wheel. \ButtonScrollBack{} means to slide your finger
  counterclockwise. Note that the wheel is sensitive, so you will need to move
  slowly at first and get a feel for how it works.

  Note that when the \ButtonHold{} switch is pushed toward the center of the \dap{},
  hold is on, and none of the other controls do anything.  Be sure
  \ButtonHold{} is off before trying to use your player.
  }
  %
  \opt{ipod1g2g}{
  The main controls on the \dap{} are a slightly indented wheel
  with a flat round button in the center, and four buttons surrounding
  it. On the 1st generation iPod, this wheel physically turns. On the
  2nd generation iPod, this wheel is touch-sensitive. Hold the \dap{} with these controls
  facing you.

  The top of the player will have the following, from left to
  right: FireWire port, headphone socket, \ButtonHold{} switch.

  The FireWire port is used to connect your \dap{} to the computer and
  to charge its battery via a wall charger.

  The button in the middle of the wheel is called \ButtonSelect{}. You can
  operate the wheel by turning it, or sliding your finger around
  it. The top is \ButtonMenu{}, the bottom is \ButtonPlay{}, the left
  is \ButtonLeft{}, and the right is \ButtonRight{}.
  When the manual says to \ButtonScrollFwd{}, it means to slide your finger
  clockwise around the wheel. \ButtonScrollBack{} means to slide your finger
  counterclockwise. Note that the wheel is sensitive, so you will need to move
  slowly at first and get a feel for how it works.

  Note that when the \ButtonHold{} switch is pushed toward the center of the \dap{},
  hold is on, and none of the other controls do anything.  Be sure
  \ButtonHold{} is off before trying to use your player.
  }
  %
  \opt{ipodnano,ipodnano2g}{
  The main controls on the \dap{} are a slightly indented wheel with a
  flat round button in the center. Hold the \dap{} with these controls on the
  top surface. There is a \ButtonHold{} switch at one end, and
  headphone and dock connector at the other; be sure the end with the
  switch is facing away from you.

  The button in the middle of the wheel is called \ButtonSelect{}. You can
  operate the wheel by pressing the top, bottom, left or right sections,
  or by sliding your finger around it.  The top is \ButtonMenu{}, the bottom is
  \ButtonPlay{}, the left is \ButtonLeft{}, and the right is \ButtonRight{}.
  When the manual says to \ButtonScrollFwd{}, it means to slide your finger
  clockwise around the wheel. \ButtonScrollBack{} means to slide your finger
  counterclockwise. Note that the wheel is sensitive, so you will need to move
  slowly at first and get a feel for how it works.

  Note that when the \ButtonHold{} switch is pushed toward the center of the \dap{},
  hold is on, and none of the other controls do anything; be sure \ButtonHold{} is
  off before trying to use your player.
  }
  %
  \opt{iriverh10,iriverh10_5gb}{
  Hold or lay the \dap{} so that the side with the scroll pad and
  LCD is facing towards you. In the centre below the lcd is the scroll pad. It
  is oriented vertically. Touching the top and bottom half of it acts as the
  \ButtonScrollUp{}  and \ButtonScrollDown{} buttons respectively. On the left
  of the scroll pad is the \ButtonLeft{} button and on the right is the
  \ButtonRight{} button.

  There are three buttons on the right hand side of the \dap{}. From top to
  bottom, they are: \ButtonRew{}, \ButtonPlay{} and \ButtonFF{}. On the left
  hand side is the \ButtonPower{} button.

  On the top panel of the \dap{}, from left to right, you can find the
  following: \ButtonHold{} switch, \opt{iriverh10}{reset pin hole, }remote port
  and headphone mini jack plug.

  On the bottom panel of the \dap{} is the data cable port.}
  %
  \opt{gigabeatf}{
  \note{The following description is for the Gigabeat F, but can also apply for the
  Gigabeat X. The Gigabeat F is slightly larger and more rectangular shaped, while the
  Gigabeat X is smaller and has a slightly tapered back.}

  Hold the \dap{} with the screen on top and the controls on the right hand side.
  Below the screen is a cross-shaped touch sensitive pad which contains the
  \ButtonUp{}, \ButtonDown{}, \ButtonLeft{} and \ButtonRight{} controls.  On the
  Gigabeat X, this pad will feel slightly raised up, while it will feel slightly
  sunken in on the Gigabeat F. On the top of the unit, from left to right, are the
  power socket, the \ButtonHold{} switch, and the headphone socket.  The
  \ButtonHold{} switch puts the \dap{} into hold mode when it is switched to the
  right of the unit. The buttons will have no effect when this is the case.

  Starting from the left hand side on the bottom of the unit, nearer to the front
  than the back, is a recessed switch which
  controls whether the battery is on or off.  When this switch is to the left,
  the battery is disconnected.  This can be used for a hard reset of the unit,
  or if the \dap{} is being placed in storage.  Next to that is a connector for
  the docking station and finally on the right hand side of the bottom of the
  unit is a mini USB socket for connecting directly to USB.

  Finally on the right hand side of the unit are some control buttons.  Going from
  the bottom of the unit to the top there is a small round \ButtonA{} buttton then a
  rocker volume switch with of the \ButtonVolDown{} button below the \ButtonVolUp{}
  button.  Above that is are two more small round buttons, the \ButtonMenu{}
  button and nearest to the top of the unit the \ButtonPower{} button, which is held
  down to turn the \dap{} on or off. If you have a Gigabeat X, these buttons are small
  metallic buttons that are place further up on the right hand side, and closer
  together. The layout is still the same, however.}
  %
  \opt{gigabeats}{
  Hold the \dap{} with the screen on top and the controls on the right hand side.
  Directly below the bottom edge of the screen are two buttons, \ButtonBack{}
  on the left and \ButtonMenu{} on the right. Below them is a cross-shaped pad
  which contains the \ButtonUp{}, \ButtonDown{}, \ButtonLeft{}, \ButtonRight{}
  and \ButtonSelect{} controls.
  On the top of the unit from left to right are the headphone socket and the
  \ButtonHold{} switch.  The \ButtonHold{} switch puts the \dap{} into
  hold mode when it is switched to the right of the unit.
  The buttons will have no effect when this is the case.

  Starting from the left hand side on the bottom of the unit, nearer to the back
  than the front, is a recessed switch which controls whether the battery is on
  or off.  When this switch is to the left, the battery is disconnected.
  This can be used for a hard reset of the unit, or if the \dap{} is being placed
  in storage.  Next to that is a mini USB socket for connecting directly to USB,
  and finally a custom connector, presumably for planned accessories which were
  never released.

  Finally on the right hand side of the unit are some control buttons and the power
  connector.  Going from the bottom of the unit to the top, there is the power
  connector socket, followed by three small round buttons, the
  \ButtonNext{} buttton, \ButtonPlay{} button, and \ButtonPrev{} button (from bottom
  to top) then a rocker volume switch with of the \ButtonVolDown{} button below the
  \ButtonVolUp{} button.  Above that is one more small round button, the \ButtonPower{}
  button, which is held down to turn the \dap{} on or off.}
  %
  \opt{mrobe100}{
  Hold the \dap{} with the black front facing you such that the m:robe writing
  is readable. Below the writing is the touch sensitive pad with the
  \ButtonMenu{}, \ButtonPlay{}, \ButtonLeft{}, \ButtonRight{} and \ButtonDisplay
  controls indicated by their symbols. The dotted center strip is devided in
  three parts: \ButtonUp{}, \ButtonSelect{} and \ButtonDown. On the top of the
  unit, on the right, is the \ButtonPower{} switch, which is held down to turn
  the \dap{} on or off.

  The \ButtonHold{} switch is located on the left of the \dap{}, below the
  headphone socket. It puts the \dap{} into hold mode when it is switched to the
  top of the unit. The buttons will have no effect when this is the case. On the
  bottom of the unit, there is a connector for the docking station or the
  proprietary USB connector for connecting directly to USB.}
  %
  \opt{iaudiom5,iaudiox5}{
  The \dap{} is curved so that the end with the screen on it is thicker than the
  other end.  Hold the \dap{} wih the thick end towards the top and the screen
  facing towards you.  Half way up the front of the unit on the right hand side
  is a four way joystick which is the \ButtonUp{}, \ButtonDown{},
  \ButtonLeft{}, and \ButtonRight{} buttons. When pressed it serves as \ButtonSelect{}.

  On the right hand side of the \dap{} from top to bottom, first there is a two
  way switch.  the \ButtonPower{} button is activated by pushing this switch up,
  and pushing this switch down until it clicks slightly will activate the
  \ButtonHold{} button.  When the switch is in this position, none of the other
  keys will have an effect.

  Below the switch is a lozenge shaped button which is the \ButtonRec{}
  button, and below that the final button on this side of the unit, the
  \ButtonPlay{} button.  Just below this is a small hole which is difficult to
  locate by touch which is the internal microphone.  At the very bottom of
  this side of the unit is the reset hole, which can be used to perform a hard
  reset by inserting a paper clip.

  On the bottom of the unit is the connector for the
  \playerman{} subpack or dock.  On the top of the unit is a charge
  indicator light, which may feel a bit like a button, but is not.

  From the top of the \dap{} on the left hand side is the headphone socket, then the
  remote connector.  Below this is a cover which protects the \opt{iaudiox5}{USB
  host connector.}\opt{iaudiom5}{USB and charging connector}.}
  %
  \opt{e200,e200v2}{
  Hold the \dap{} with the turning wheel at the front and bottom.  On the bottom left
  of the front of the \dap{} is a raised round button, the \ButtonPower{} button.
  Above and to the left of this, on the outside of the turning wheel are four
  buttons.  These are the \ButtonUp{}, \ButtonDown{}, \ButtonLeft{} and
  \ButtonRight{} buttons.  Inside the wheel is the \ButtonSelect{} button.  Turning
  the wheel to the right activates the \ButtonScrollFwd{} function, and to the
  left, the \ButtonScrollBack{} function.

  On the right of the unit is a slot for inserting flash cards.  On the bottom is
  the connector for the USB cable.  On the left is the \ButtonRec{} button, and
  on the top, there is the headphone socket to the right, and the \ButtonHold{}
  switch.  Moving this switch to the right activates hold mode in which none of the
  other buttons have any effect.  Just to the left of the \ButtonHold{} switch is a
  small hole which contains the internal microphone.}
  %
  \opt{c200,c200v2}{
  Hold the \dap{} with the buttons on the right and the screen on the left. On
  the right side of the unit, there is a series of four connected buttons that
  form a square. The four sides of the square are the \ButtonUp{},
  \ButtonDown{}, \ButtonLeft{} and \ButtonRight{} buttons, respectively. Inside
  the square formed by these four buttons is the \ButtonSelect{} button. At the
  bottom right corner of the square is a small separate button, the
  \ButtonPower{} button.

  Moving clockwise around the outside of the unit, on the top are the \ButtonVolUp{}
  and \ButtonVolDown{} buttons, which control the volume of playback. The buttons can
  be distinguished by a sunken triangle on the \ButtonVolDown{} button, and a
  raised triangle on the \ButtonVolUp{} button. To the right of
  the volume buttons on the top of the unit is the slot for inserting flash
  memory cards. On the right side of the unit is the connector for the USB
  cable. At center of the bottom of the \dap{} is the \ButtonRec{} button. To
  the left of the \ButtonRec{} button is the \ButtonHold{} switch. Moving this
  switch to the right activates hold mode, in which none of the other buttons
  have any effect. On the lower left side of the unit is the headphone socket.
  Immediately above the headphone socket is a lanyard loop and the microphone.
  }
  %
  \opt{fuze,fuzev2}{
  Hold the \dap{} with the controls on the bottom and the screen on the top. The main
  controls are a scroll wheel with four clickable points and a button in the centre; pressing
  this centre button functions as \ButtonSelect{}. Going clockwise from the top, the clickable
  points on the wheel are the \ButtonUp{}, \ButtonRight{}, \ButtonDown{}, and \ButtonLeft{}
  buttons. Turning the wheel clockwise is \ButtonScrollFwd{}, and turning it counter-clockwise
  is \ButtonScrollBack{}. Immediately above and to the right of the wheel is the \ButtonHome{}
  button.

  On the lower left of the unit is a slot for inserting microSD cards. Immediately below that is
  the opening for the microphone.

  On the bottom of the unit is the connector for connecting a USB cable and the headphone socket.
  On the lower right hand side of the unit is a two-way switch. Pressing this switch up acts as
  \ButtonPower{}, and clicking it down until it locks acts as the \ButtonHold{} switch. When the
  \ButtonHold{} switch is on, none of the other buttons have any effect.
  }
  %
  \opt{clipplus,clipv1,clipv2,clipzip}{
  Hold the \dap{} with the controls on the bottom and the screen on the top. The main
  controls are a four-way pad with a button in the centre; pressing this centre button
  functions as \ButtonSelect{}. Going clockwise from the top, the four-way pad contains
  the \ButtonUp{}, \ButtonRight{}, \ButtonDown{}, and \ButtonLeft{} buttons.
  Immediately above and to the \nopt{clipzip}{right}\opt{clipzip}{left} of the four-way
  pad is the \ButtonHome{} button.
  }
  %
  \opt{clipplus,clipzip}{
  The \ButtonPower{} button is on the top of the \dap{}\opt{clipplus}{, towards the right side.}

  At the bottom of the right side of the \dap{} is a slot for microSD cards.
  Above this slot on the right side is the headphone socket.

  On the left hand panel is a two-way button that acts as \ButtonVolDown{} when
  pressed on the bottom, and \ButtonVolUp{} when pressed on the top. Immediately
  above the switch is a mini-USB port to connect the \dap{} to a computer.

  }
  %
  \opt{clipv1,clipv2}{
  On the left hand panel is a two way switch. Pressing this switch up acts as
  \ButtonPower{}, and clicking it down until it locks acts as the \ButtonHold{}
  switch. When the \ButtonHold{} switch is on, none of the other buttons have any
  effect. Immediately above the switch is a mini-USB port to connect the \dap{} to
  a computer.

  On the right hand panel is a two-way button that acts as \ButtonVolDown{} when
  pressed on the bottom, and \ButtonVolUp{} when pressed on the top. Immediately
  above this button is the headphone socket.
  }
  %
  \opt{vibe500}{
  Hold or lay the \dap{} so that the side with the controls and
  LCD is facing towards you. Below the LCD is the touch sensitive pad with the \ButtonMenu{},
  \ButtonPlay{}, \ButtonLeft{}, \ButtonRight{} controls and the scroll pad in the centre. The
  scroll pad is oriented vertically between the \ButtonOK{} and \ButtonCancel{} buttons.
  Sliding a finger up or down the scroll pad acts as \ButtonUp{} and \ButtonDown{} respectively.
  Note that the scroll pad is sensitive, so you will need to move
  slowly at first and get a feel for how it works.

  There are two buttons on the right hand side of the \dap{}: \ButtonPower{} on the top and
  \ButtonRec{} underneath. Under these buttons, from top to bottom you can find: USB connector,
  power connector and the reset hole if you need to perform a hardware reset.

  The \ButtonHold{} switch is located on the left hand side of the \dap{}. Note that when the
  \ButtonHold{} switch is moved towards the top of the \dap{}, hold is turned on and all the
  other controls are disabled. Be sure \ButtonHold{} is off before trying to use your player.

  On the top on the \dap{} is the internal microphone on the left and the line-in socket on the
  right, near the headphone socket.}
  %
  \opt{samsungyh820}{
  Hold or lay the \dap{} so that the side with the controls and
  LCD is facing towards you. Directly below the bottom edge of the screen are three buttons:
  \ButtonRew{} on the left, \ButtonPlay{} in the middle and \ButtonFF{} on the right. Below them
  is a four-way pad which contains the \ButtonDown{}, \ButtonUp{}, \ButtonLeft{} and
  \ButtonRight{} controls.

  At the top of the right hand side of the \dap{} is the \ButtonRec{} button.

  On the top panel of the \dap{}, from left to right, you can find the following: headphone
  socket, line-in socket, internal microphone, and the \ButtonHold{} switch. Note that when the
  \ButtonHold{} switch is moved towards the center of the \dap{}, hold is turned on and all the
  other controls are disabled. Be sure \ButtonHold{} is off before trying to use your player.

  At the top of the back side of the player, just under the \ButtonHold{} button is the reset
  hole, if you need to perform a hardware reset.

  The USB/dock connector that is used to connect your \dap{} to your computer is on
  the bottom panel of the \dap{}.
  }
  %
  \opt{samsungyh920,samsungyh925}{
  Hold or lay the \dap{} so that the side with the controls and
  LCD is facing towards you. Below the LCD is a four-way pad with the \ButtonDown{},
  \ButtonUp{}, \ButtonLeft{} and \ButtonRight{} buttons.

  There are three buttons at the top of the right hand side of the \dap{}: \ButtonFF{} on the top,
  \ButtonPlay{} in the middle and \ButtonRew{} underneath. Below these buttons is the \ButtonRec{}
  switch. Rockbox doesn't take note of the actual \emph{position} of the switch, but reacts to a
  \emph{switching movement} like pressing a regular button.

  On the top panel of the \dap{}, from left to right, you can find the following: headphone/remote
  socket, line-in socket, internal microphone, and the \ButtonHold{} switch. Note that when the
  \ButtonHold{} switch is moved towards the center of the \dap{}, hold is turned on and all the
  other controls are disabled. Be sure \ButtonHold{} is off before trying to use your player.

  At the top of the back side of the player, just under the \ButtonHold{} button is the reset hole,
  if you need to perform a hardware reset.

  The USB/dock connector that is used to connect your \dap{} to your computer is on
  the bottom panel of the \dap{}.
  }
  %
}

\subsection{Turning the \dap{} on and off}
\opt{cowond2}{Rockbox has a dual-boot feature with the original firmware being
  the default.\\}
To turn on and off your Rockbox enabled \dap{} use the following keys:
    \begin{btnmap}
      \opt{IRIVER_H100_PAD,IRIVER_H300_PAD}{\ButtonOn}%
      \opt{MPIO_HD200_PAD,MPIO_HD300_PAD,SAMSUNG_YH92X_PAD,SAMSUNG_YH820_PAD}%
          {Long \ButtonPlay}%
      \opt{IPOD_4G_PAD}{\ButtonMenu{} / \ButtonSelect}%
      \opt{IPOD_3G_PAD}{\ButtonMenu{} / \ButtonPlay}%
      \opt{IAUDIO_X5_PAD,IRIVER_H10_PAD,SANSA_E200_PAD,SANSA_C200_PAD,ONDA_VX777_PAD%
          ,GIGABEAT_PAD,MROBE100_PAD,GIGABEAT_S_PAD,sansaAMS,PBELL_VIBE500_PAD%
          ,SANSA_FUZEPLUS_PAD,XDUOO_X3_PAD,AIGO_EROSQ_PAD%
          }{\ButtonPower}%
      \opt{COWON_D2_PAD} {\ButtonPower{}, then \ButtonHold}%
      \opt{ONDA_VX777_PAD} {\ButtonPower{}}%
      \opt{AGPTEK_ROCKER_PAD}{\ButtonPower{}}%
          &
      \opt{HAVEREMOTEKEYMAP}{
          \opt{IRIVER_RC_H100_PAD}{\ButtonRCOn}%
          \opt{IAUDIO_RC_PAD}{\ButtonRCPlay}
          &}

      Start Rockbox
          \\

      \opt{IRIVER_H100_PAD,IRIVER_H300_PAD}{Long \ButtonOff}%
      \opt{MPIO_HD200_PAD,MPIO_HD300_PAD,SAMSUNG_YH92X_PAD,SAMSUNG_YH820_PAD}%
          {Long \ButtonPlay}%
      \opt{IPOD_4G_PAD,IPOD_3G_PAD}{Long \ButtonPlay}%
      \opt{IAUDIO_X5_PAD,IRIVER_H10_PAD,SANSA_E200_PAD,SANSA_C200_PAD%
          ,GIGABEAT_PAD,MROBE100_PAD,GIGABEAT_S_PAD,sansaAMS,COWON_D2_PAD%
          ,PBELL_VIBE500_PAD,ONDA_VX777_PAD,SANSA_FUZEPLUS_PAD,XDUOO_X3_PAD,AIGO_EROSQ_PAD%
          }{Long \ButtonPower}%
      \opt{AGPTEK_ROCKER_PAD}{Long \ButtonPower{}}%
          &
      \opt{HAVEREMOTEKEYMAP}{
          \opt{IRIVER_RC_H100_PAD}{Long \ButtonRCStop}%
          \opt{IAUDIO_RC_PAD}{Long \ButtonRCPlay}
          &}

      Shutdown Rockbox
          \\
    \end{btnmap}

\label{ref:Safeshutdown}On shutdown, Rockbox automatically saves its settings.

\opt{IRIVER_H100_PAD,IRIVER_H300_PAD,IAUDIO_X5_PAD,SANSA_E200_PAD%
  ,SANSA_C200_PAD,IRIVER_H10_PAD,IPOD_4G_PAD,GIGABEAT_PAD}{%
  If you have problems with your settings, such as accidentally having
  set the colours to black on black, they can be reset at boot time.  See
  the Reset Settings in \reference{ref:manage_settings_menu} for details.
}%

\opt{GIGABEAT_PAD,IPOD_4G_PAD,SANSA_E200_PAD%
,SANSA_C200_PAD,IAUDIO_X5_PAD,IAUDIO_M5_PAD,IPOD_3G_PAD}{%
  In the unlikely event of a software failure, hardware poweroff or reset can be
  performed by holding down
  \opt{GIGABEAT_PAD}{the battery switch}\opt{IPOD_4G_PAD}
  {\ButtonMenu{} and \ButtonSelect{} simultaneously}%
  \opt{IPOD_3G_PAD}{\ButtonMenu{} and \ButtonPlay{} simultaneously}%
  \opt{SANSA_E200_PAD,SANSA_C200_PAD,IAUDIO_X5_PAD,IAUDIO_M5_PAD}
  {\ButtonPower} until the \dap{} shuts off or reboots.
}%
\opt{IRIVER_H100_PAD,IRIVER_H300_PAD,IAUDIO_M3_PAD,IRIVER_H10_PAD,MROBE100_PAD
    ,PBELL_VIBE500_PAD,MPIO_HD200_PAD,MPIO_HD300_PAD,SAMSUNG_YH92X_PAD%
    ,SAMSUNG_YH820_PAD,XDUOO_X3_PAD}{%
  In the unlikely event of a software failure, a hardware reset can be
  performed by inserting a paperclip gently into the Reset hole.
}%

\nopt{gigabeatf,iaudiom3,iaudiom5,iaudiox5}
  {
  \subsection{Starting the original firmware}
  \label{ref:Dualboot}
  \opt{ipod4g,ipodcolor,ipodvideo,ipodnano,ipodnano2g,ipodmini}
    {
    Rockbox has a dual-boot feature. To boot into the original firmware, shut
    down the device as described above. Turn on the \ButtonHold{} switch
    immediately after turning the player on. The Apple logo will
    display for a few seconds as Rockbox loads the original firmware.

    You can also load the original firmware by shutting down the device,
    then clicking the \ButtonHold{} switch on and connecting the iPod
    to your computer.

    Regardless of which method you use to boot to the original firmware, you can
    return to Rockbox by pressing and holding \ButtonMenu{} and \ButtonSelect{}
    simultaneously until the player hard resets.
    }

  \opt{ipod1g2g,ipod3g}
    {
    Rockbox has a dual-boot feature. To boot into the original firmware, shut
    down the device as described above. Turn on the \ButtonHold{} switch
    immediately after turning the player on. The Apple logo will
    display for a few seconds as Rockbox loads the original firmware.

    You can also load the original firmware by shutting down the device,
    then clicking the \ButtonHold{} switch on and connecting the iPod
    to your computer.

    Regardless of which method you use to boot to the original firmware, you can
    return to Rockbox by pressing and holding \ButtonMenu{} and \ButtonPlay{}
    simultaneously until the player hard resets.
    }

  \opt{iriverh100,iriverh300}
    {
    Rockbox has a dual-boot feature. To boot into the original firmware,
    when the \dap{} is turned off, press and hold the \ButtonRec{} button,
    and then press the \ButtonOn{} button.
    }
  \opt{fuzeplus}
    {
    Rockbox has a dual-boot feature. To boot into the original firmware,
    when the \dap{} is turned off, press and hold the \ButtonVolDown{} button,
    and then press and hold the \ButtonPower{} button while keeping the
    \ButtonVolDown{} button pressed. After 5 to 10 seconds the original
    firmware should boot.

    It is also possible to connect your \dap{} to your computer using the
    original firmware. To do so you may press and hold the \ButtonVolDown{}
    button and connect your device to the computer while keeping the
    \ButtonVolDown{} button pressed. After 5 to 10 seconds the original
    firmware should boot into USB mode.
    }
  \opt{mpiohd200,mpiohd300}
    {
    Rockbox has a dual-boot feature. To boot into the original firmware,
    when the \dap{} is turned off, press and hold the \ButtonRec{} button,
    and then press the \ButtonPlay{} button. This will bring you to the
    short menu where you can choose among: Boot Rockbox, Boot MPIO firmware
    and Shutdown. Select the option you need with \ButtonRew{} and \ButtonFF{}
    and confirm with long \ButtonPlay{}.
    }
  \opt{iriverh10,iriverh10_5gb}
    {
    Rockbox has a dual-boot feature. It loads the original firmware from
    the file \fname{/System/OF.mi4}. To boot into the original firmware,
    press and hold the \ButtonLeft{} button while turning on the player.
    \note{The iriver firmware does not shut down properly when you turn it off,
    it only goes to sleep. To get back into Rockbox when exiting from the
    iriver firmware, you will need to reset the player by \opt{iriverh10}{%
    inserting a pin in the reset hole}\opt{iriverh10_5gb}{removing and
    reinserting the battery}.}
    }

  \opt{sansa,sansaAMS}
    {
    Rockbox has a dual-boot feature. To boot into the original firmware,
    press and hold the \ButtonLeft{} button while turning on the player.
    }

  \opt{clipv2,fuzev2,clipplus}
    {
        \note{Rockbox does not boot into the original firmware when powered by
        a USB connection. Older versions of Rockbox do not provide USB support.
        If you have such a version installed you need to manually boot into the
        original firmware for data transfer via USB.}
    }

  \opt{mrobe100}
    {
    Rockbox has a dual-boot feature. It loads the original firmware from
    the file \fname{/System/OF.mi4}. To boot into the original firmware,
    when the \dap{} is turned off, press the \ButtonPower{} button once and then
    a second time when the m:robe bootlogo (the headphone) appears. Hold the
    \ButtonPower{} button until you see the ``Loading original firmware...''
    message on the screen.
    }

  \opt{gigabeats}
    {
    Rockbox has a dual-boot feature. To boot into the original firmware,
    turn the \ButtonHold{} switch on just after turning on the \dap{}.
    To return to Rockbox, shutdown the \dap{}, then turn the battery switch
    on the bottom off then on again. Rockbox should now start.
    }

  \opt{cowond2}
    {
    Use \ButtonPower{} to boot the original \playerman{} firmware.
    }

  \opt{vibe500}
    {
    Rockbox has a dual-boot feature where it is possible to load the original firmware from
    the file \fname{/System/OF.mi4}. To boot into the original firmware press and release
    \ButtonPower{} and then immediately after the backlight turns on, press the \ButtonOK{}
    button and keep it pressed until the original firmware starts.
    }

  \opt{samsungyh}
    {
    Rockbox has a dual-boot feature. It loads the original firmware from
    the file \fname{/System/OF.mi4}. To boot into the original firmware, press and hold
    for awhile the \ButtonPlay{} button and then immediately after the Samsung logo appears,
    press the \ButtonLeft{} button and keep it pressed until the original firmware starts.
    }

  \opt{ondavx777}
    {
    Rockbox has a dual-boot feature where it is possible to load the original firmware from
    the file \fname{/SD/ccpmp.bin}. To boot into the original firmware press and release
    \ButtonPower{} immediately after the Rockbox Logo appear on the screen.
    }

  \opt{xduoox3}
    {
    Rockbox has a dual-boot feature. To boot into the original firmware,
    when the \dap{} is turned off, set the \ButtonLock{} switch to locked,
    and then press the \ButtonPower{} button.
    }

  \opt{fiiom3k,shanlingq1,erosqnative}
    {
    Rockbox has a dual-boot feature. To boot into the original firmware,
    hold \ActionBootOFPlayer{} when powering on the \dap{}.

    \nopt{erosqnative}{
      You can trigger a normal \playerman{} firmware update by holding
      \ActionBootOFRecovery{} when powering on the \dap{}.
      \warn{Updating the original firmware will \textbf{erase} the Rockbox
      bootloader.}
    }

    \subsection{Entering the recovery menu}
    You can access the Rockbox bootloader's ``recovery menu'' by holding
    \ActionBootRecoveryMenu{}. This menu can be used to connect your \dap{}
    over USB to transfer files, update the Rockbox bootloader, or revert to a
    bootloader you've previously backed up.
    }

  }
\subsection{Putting music on your \dap{}}

\opt{usb_hid}{
\note{Due to a bug in some OS X versions, the \dap{} can not be mounted, unless
    the USB HID feature is disabled. See \reference{ref:USB_HID} for more
    information.\newline
}
}

With the \dap{} connected to the computer as an MSC/UMS device (like a
USB Drive), music files can be put on the player via any standard file
transfer method that you would use to copy files between drives (e.g. Drag-and-Drop).
Files may be placed wherever you like on the \dap{}, but it is strongly
suggested \emph{NOT} to put them in the \fname{/.rockbox} folder and instead
put them in any other folder, e.g. \fname{/}, \fname{/music} or \fname{/audio}.
The default directory structure that is assumed by some parts of Rockbox
\opt{albumart}{%
    (album art searching, and missing-tag fallback in some WPSes) uses the
    parent directory of a song as the Album name, and the parent directory of
    that folder as the Artist name. WPSes may display information incorrectly if
    your files are not properly tagged, and you have your music organized in a
    way different than they assume when attempting to guess the Artist and Album
    names from your filetree. See \reference{ref:album_art} for the requirements
    for Album Art to work properly.
}%
\nopt{albumart}{%
    (missing-tag fallback in some WPSes) uses the parent directory of a song
    as the Album name, and the parent directory of that folder as the Artist
    name. WPSes may display
    information incorrectly if your files are not properly tagged, and you have
    your music organized in a way different than they assume when attempting to
    guess the Artist and Album names from your filetree.
}%
    See \reference{ref:Supportedaudioformats} for a list of supported audio
    formats.

\subsection{The first contact}

After you have first started the \dap{}, you'll be presented by the
\setting{Main Menu}. From this menu you can reach every function of Rockbox,
for more information (see \reference{ref:main_menu}). To browse the files
on your \dap{}, select \setting{Files} (see \reference{ref:file_browser}), and to
browse in a view that is based on the meta-data\footnote{ID3 Tags, Vorbis
comments, etc.} of your audio files, select \setting{Database} (see
\reference{ref:database}).

\subsection{Basic controls}
When browsing files and moving through menus you usually get a list view
presented. The navigation in these lists are usually the same and should be
pretty intuitive.
In the tree view use \ActionStdNext{} and \ActionStdPrev{} to move around
the selection. Use \ActionStdOk{} to select an item. \opt{wheel_acceleration}{
Note that the scroll speed is accelerating the faster you rotate the wheel.}
When browsing the file system selecting an audio file plays it. The view
switches to the ``While playing screen'', usually abbreviated as ``WPS'' (see
\reference{ref:WPS}. The dynamic playlist gets replaced with the contents of
the current directory. This way you can easily treat directories as playlists.
The created dynamic playlist can be extended or modified while playing. This is
also known as ``on-the-fly playlist''.
To go back to the \setting{File Browser} stop the playback with the
\ActionWpsStop{} button or return to the file browser while keeping playback
running using \ActionWpsBrowse{}.
In list views you can go back one step with \ActionTreeParentDirectory.

\subsection{Basic concepts}
\subsubsection{Playlists}
Rockbox is playlist oriented. This means that every time you play an audio file,
a so-called ``dynamic playlist'' is generated, unless you play a saved
playlist. You can modify the dynamic playlist while playing and also save
it to a file. If you do not want to use playlists you can simply play your
files directory based.
Playlists are covered in detail in \reference{ref:working_with_playlists}.

\subsubsection{Menu}
From the menu you can customise Rockbox. Rockbox itself is very customisable.
Also there are some special menus for quick access to frequently used
functions.

\subsubsection{Context Menu}
Some views, especially the file browser and the WPS have a context menu.
From the file browser this can be accessed with \ActionStdContext{}.
The contents of the context menu vary, depending on the situation it gets
called. The context menu itself presents you with some operations you can
perform with the currently highlighted file. In the file browser this is
the file (or directory) that is highlighted by the cursor. From the WPS this is
the currently playing file. Also there are some actions that do not apply
to the current file but refer to the screen from which the context menu
gets called. One example is the playback menu, which can be called using
the context menu from within the WPS.

\section{Customising Rockbox}
Rockbox' User Interface can be customised using ``Themes''. Themes usually
only affect the visual appearance, but an advanced user can create a theme
that also changes various other settings like file view, LCD settings and
all other settings that can be modified using \fname{.cfg} files. This topic
is discussed in more detail in \reference{ref:manage_settings}.
The Rockbox distribution comes with some themes that should look nice on
your \dap{}.

\note{Some of the themes shipped with Rockbox need additional
fonts from the fonts package, so make sure you installed them.
Also, if you downloaded additional themes from the Internet make sure you
have the needed fonts installed as otherwise the theme may not display
properly.}

  \opt{usb_power}{
    \section{USB Charging}
    Your \dap{} will automatically charge when connected to USB. By default
    Rockbox will connect in mass storage mode to transfer files, but you can
    prevent this by holding down any button while plugging in the USB cable,
    or by changing the \setting{USB Mode} setting to \setting{Charge Only}.
    \nopt{fuzeplus}{
    \note{Be aware that holding a button may still perform its normal function,
    so it is recommended to use a button without harmful side effects, such as
    \ActionStdUsbCharge{}.}
    }
  }

% $Id$ %
\chapter{Browsing and playing}
\section{\label{ref:file_browser}File Browser}
\screenshot{rockbox_interface/images/ss-file-browser}{The file browser}{}
Rockbox lets you browse your music in either of two ways. The 
\setting{File Browser} lets you navigate through the files and directories on 
your \dap, entering directories and executing the default action on each file.
To help differentiate files, each file format is displayed with an icon. 

The \setting{Database Browser}, on the other hand, allows you to navigate 
through the music on your player using categories like album, artist, genre,
etc.

You can select whether to browse using the \setting{File Browser} or the
\setting{Database Browser} by selecting either \setting{Files} or
\setting{Database} in the \setting{Main Menu}.
If you choose the \setting{File Browser}, the \setting{Show Files} setting
lets you select what types of files you wish to view. See
\reference{ref:ShowFiles} for more information on the \setting{Show Files}
setting.

\note{The \setting{File Browser} allows you to manipulate your files in ways
that are not available within the \setting{Database Browser}. Read more about
\setting{Database} in \reference{ref:database}. The remainder of this section
deals with the \setting{File Browser}.}

\opt{iriverh10,iriverh10_5gb}{\note{
If your \dap{} is a MTP model, the Music directory where all your music is stored
may be hidden in the \setting{File Browser}. This may be fixed by either
changing its properties (on a computer) to not hidden, or by changing
the \setting{Show Files} setting to all.
}}

\subsection{\label{ref:controls}File Browser Controls}
\begin{btnmap}
      \ActionStdPrev{}/\ActionStdNext{}
      \opt{HAVEREMOTEKEYMAP}{& \ActionRCStdPrev{}/\ActionRCStdNext{}}
         & Go to previous/next item in list. If you are on the first/last 
           entry, the cursor will wrap to the last/first entry.\\
      %
      \opt{IRIVER_H100_PAD,IRIVER_H300_PAD}
        {
          \ButtonOn+\ButtonUp{}/ \ButtonDown
          \opt{HAVEREMOTEKEYMAP}{&
            \opt{IRIVER_RC_H100_PAD}{\ButtonRCSource{}/ \ButtonRCBitrate}
          }
          & Move one page up/down in the list.\\
        }
      \opt{IRIVER_H10_PAD}
        {
          \ButtonRew{}/ \ButtonFF
          & Move one page up/down in the list.\\
        }
      %
      \ActionTreeParentDirectory
      \opt{HAVEREMOTEKEYMAP}{& \ActionRCTreeParentDirectory}
      & Go to the parent directory.\\
      %
      \ActionTreeEnter
      \opt{HAVEREMOTEKEYMAP}{& \ActionRCTreeEnter}
      & Execute the default action on the selected file or enter a
        directory.\\
      %
      \ActionTreeWps 
      \opt{HAVEREMOTEKEYMAP}{& \ActionRCTreeWps}
         & If there is an audio file playing, return to the
         \setting{While Playing Screen} (WPS) without stopping playback.\\
      %
      \nopt{player,SANSA_C200_PAD,erosqnative}%
        {%
          \ActionTreeStop 
          \opt{HAVEREMOTEKEYMAP}{& \ActionRCTreeStop}
          & Stop audio playback.\\%
        }%
      %
      \ActionStdContext{}
      \opt{HAVEREMOTEKEYMAP}{& \ActionRCStdContext}
      & Enter the \setting{Context Menu}.\\
      %
      \ActionStdMenu{}
      \opt{HAVEREMOTEKEYMAP}{& \ActionRCStdMenu}
      & Enter the \setting{Main Menu}.\\
      %
      \nopt{erosqnative}{
        \opt{quickscreen}{
          \ActionStdQuickScreen
          \opt{HAVEREMOTEKEYMAP}{& \ActionRCStdQuickScreen}
          & Switch to the \setting{Quick Screen}
          (see \reference{ref:QuickScreen}). \\
        }
      }
      %
      \opt{SANSA_E200_PAD}{
        \ActionStdRec & Switch to the \setting{Recording Screen}.\\
      %
      }
      \nopt{touchscreen}{\opt{hotkey}{
        \ActionTreeHotkey
            &
        \opt{HAVEREMOTEKEYMAP}{
            &}
        Activate the \setting{Hotkey} function
        (see \reference{ref:Hotkeys}).
            \\
      }}
\end{btnmap}

\subsection{\label{ref:Contextmenu}\label{ref:PartIISectionFM}Context Menu}
\screenshot{rockbox_interface/images/ss-context-menu}{The Context Menu}{}

The \setting{Context Menu} allows you to perform certain operations on files or 
directories.  To access the \setting{Context Menu}, position the selector over a file 
or directory and access the context menu with \ActionStdContext{}.\\

\note{The \setting{Context Menu} is a context sensitive menu.  If the 
\setting{Context Menu} is invoked on a file, it will display options available 
for files.  If the \setting{Context Menu} is invoked on a directory, 
it will display options for directories.\\}

The \setting{Context Menu} contains the following options (unless otherwise noted, 
each option pertains both to files and directories):

\begin{description}
\item [View.]
  Displays the contents of the selected playlist file.
\item [Current Playlist.]
  Enters the \setting{Current Playlist Submenu} (see \reference{ref:currentplaylist_submenu}).
\item [Playlist Catalogue.]
  Enters the \setting{Playlist Catalogue Submenu} (see 
  \reference{ref:playlist_catalogue}).
\item [Rename.]
  This function lets the user modify the name of a file or directory.
\item [Cut.]
  Copies the name of the currently selected file or directory to the clipboard
  and marks it to be `cut'.
\item [Copy.]
  Copies the name of the currently selected file or directory to the clipboard
  and marks it to be `copied'.
\item [Paste.]
  Only visible if a file or directory name is on the clipboard. When selected
  it will move or copy the clipboard to the current directory.
\item [Delete.]
  Deletes the currently selected file. This option applies only to files, and
  not to directories. Rockbox will ask for confirmation before deleting a file.
  Press \ActionYesNoAccept{}
  to confirm deletion or any other key to cancel.
\item [Delete Directory.]
  Deletes the currently selected directory and all of the files and subdirectories
  it may contain. Deleted directories cannot be recovered. Use this feature with
  caution!
\opt{lcd_non-mono}{
\item [Set As Backdrop.]
  Set the selected \fname{bmp} file as background image. The bitmaps need to meet the
  conditions explained in \reference{ref:LoadingBackdrops}.
}
\item [Open with.]
  Runs a viewer plugin on the file. Normally, when a file is selected in Rockbox,
  Rockbox automatically detects the file type and runs the appropriate plugin.
  The \setting{Open With} function can be used to override the default action and
  select a viewer by hand.
  For example, this function can be used to view a text file
  even if the file has a non-standard extension (i.e., the file has an extension
  of something other than \fname{.txt}). See \reference{ref:Viewersplugins}
  for more details on viewers.
\item [Create Directory.]
  Create a new directory in the current directory on the disk.
\item [Properties.]
  Shows properties such as size and the time and date of the last modification
  for the selected file. If used on a directory, the number of files and
  subdirectories will be shown, as well as the total size.
\opt{recording}{
  \item [Set As Recording Directory.]
    Save recordings in the selected directory.
}
\item [\label{ref:StartFileBrowserHere}Start File Browser Here.]
  This option allows users to set the currently selected directory as the default
  start directory for the file browser. This option is not available for files.
  \note{If you have \setting{Auto-Change Directory} and
  \setting{Constrain Auto-Change} enabled, the directories returned will
  be constrained to the directory you have chosen here and those below it.
  See \reference{ref:ConstrainAutoChange}}
\item [Add to Shortcuts.]
  Adds a link to the selected item in the \fname{shortcuts.link} file.
  If the file does not already exist it will be created in the root directory.
  Note that if you create a shortcut to a file, Rockbox will not open it upon
  selecting, but simply bring you to its location in the \setting{File Browser}.
\end{description}

\subsection{\label{sec:virtual_keyboard}Virtual Keyboard}
\screenshot{rockbox_interface/images/ss-virtual-keyboard}{The virtual keyboard}{}
This is the virtual keyboard that is used when entering text in Rockbox, for 
example when renaming a file or creating a new directory.
The virtual keyboard can be easily changed by making a text file
with the required layout. More information on how to achieve this can be found
on the Rockbox website at \wikilink{LoadableKeyboardLayouts}.

\opt{morse_input}{
  Also you can switch to Morse code input mode by changing the
  \setting{Use Morse Code Input} setting%
  \opt{IRIVER_H100_PAD,IRIVER_H300_PAD,IPOD_4G_PAD,IPOD_3G_PAD,IRIVER_H10_PAD%
      ,GIGABEAT_PAD,GIGABEAT_S_PAD,MROBE100_PAD,SANSA_E200_PAD,PBELL_VIBE500_PAD%
      ,SANSA_FUZEPLUS_PAD,SAMSUNG_YH92X_PAD,SAMSUNG_YH820_PAD}
    { or by pressing \ActionKbdMorseInput{} in the virtual keyboard}%
  .}

% no "Actions" yet in the Player's virtual keyboard

\note{When the cursor is on the input line, \ActionKbdSelect{} deletes the preceding character}

\begin{btnmap}
    \opt{IRIVER_H100_PAD,IRIVER_H300_PAD,GIGABEAT_PAD,GIGABEAT_S_PAD%
        ,MROBE100_PAD,SANSA_E200_PAD,SANSA_FUZE_PAD,SANSA_C200_PAD,SANSA_FUZEPLUS_PAD%
        ,SAMSUNG_YH820_PAD}{
        \ActionKbdCursorLeft{} / \ActionKbdCursorRight
            &
        \opt{HAVEREMOTEKEYMAP}{\ActionRCKbdCursorLeft{} / \ActionRCKbdCursorRight
            &}
        Move the line cursor within the text line.
            \\
        %
        \ActionKbdBackSpace
            &
        \opt{HAVEREMOTEKEYMAP}{
            &}
        Delete the character before the line cursor.
            \\
    }%
    \ActionKbdLeft{} / \ActionKbdRight
        &
    \opt{HAVEREMOTEKEYMAP}{\ActionRCKbdLeft{} / \ActionRCKbdRight
        &}
    Move the cursor on the virtual keyboard.
    If you move out of the picker area, you get the previous/next page of
    characters (if there is more than one).
        \\
    %
    \ActionKbdUp{} / \ActionKbdDown
        &
    \opt{HAVEREMOTEKEYMAP}{\ActionRCKbdUp{} / \ActionRCKbdDown
        &}
    Move the cursor on the virtual keyboard.
    If you move out of the picker area you get to the line edit mode.
        \\
    %
    \nopt{IPOD_3G_PAD,IPOD_4G_PAD,IRIVER_H10_PAD,PBELL_VIBE500_PAD%
         ,SANSA_FUZEPLUS_PAD,SAMSUNG_YH92X_PAD,SAMSUNG_YH820_PAD}{
        \ActionKbdPageFlip
            &
        \opt{HAVEREMOTEKEYMAP}{\ActionRCKbdPageFlip
            &}
        Flip to the next page of characters (if there is more than one).
            \\
    }
    %
    \ActionKbdSelect
        &
    \opt{HAVEREMOTEKEYMAP}{\ActionRCKbdSelect
        &}
    Insert the selected keyboard letter at the current line cursor position.
        \\
    %
    \ActionKbdDone
        &
    \opt{HAVEREMOTEKEYMAP}{\ActionRCKbdDone
        &}
    Exit the virtual keyboard and save any changes.
        \\
    %
    \ActionKbdAbort
        &
    \opt{HAVEREMOTEKEYMAP}{\ActionRCKbdAbort
        &}
    Exit the virtual keyboard without saving any changes.
        \\
% to be done - create a separate section for morse imput and update the info
      \opt{morse_input}{
        \opt{IRIVER_H100_PAD,IRIVER_H300_PAD,GIGABEAT_PAD,GIGABEAT_S_PAD,MROBE100_PADD%
            ,SANSA_E200_PA,IPOD_4G_PAD,IPOD_3G_PAD,IRIVER_H10_PAD,PBELL_VIBE500_PAD%
            ,SAMSUNG_YH92X_PAD,SAMSUNG_YH820_PAD}{
          \ActionKbdMorseInput
          \opt{HAVEREMOTEKEYMAP}{& \ActionRCKbdMorseInput}
          & Toggle keyboard input mode and Morse code input mode. \\}
        %
        \ActionKbdMorseSelect
        \opt{HAVEREMOTEKEYMAP}{& \ActionRCKbdMorseSelect}
        & Tap to select a character in Morse code input mode. \\
      } 
\end{btnmap}

% $Id$ %
\section{\label{ref:database}Database}

\subsection{Introduction}
This chapter describes the Rockbox music database system. Using the information
contained in the tags (ID3v1, ID3v2, Vorbis Comments, Apev2, etc.) in your
audio files, Rockbox builds and maintains a database of the music
files on your player and allows you to browse them by Artist, Album, Genre, 
Song Name, etc.  The criteria the database uses to sort the songs can be completely
 customised. More information on how to achieve this can be found on the Rockbox
 website at \wikilink{DataBase}. 

\subsection{Initializing the Database}
The first time you use the database, Rockbox will scan your disk for audio files.
This can take quite a while depending on the number of files on your \dap{}.
This scan happens in the background, so you can choose to return to the
Main Menu and continue to listen to music.
If you shut down your player, the scan will continue next time you turn it on.
After the scan is finished you may be prompted to restart your \dap{} before
you can use the database.

\subsubsection{Ignoring Directories During Database Initialization}

You may have directories on your \dap{} whose contents should not be added
to the database. Placing a file named \fname{database.ignore} in a directory
will exclude the files in that directory and all its subdirectories from
scanning their tags and adding them to the database. This will speed up the
database initialization.

If a subdirectory of an `ignored' directory should still be scanned, place a
file named \fname{database.unignore} in it. The files in that directory and
its subdirectories will be scanned and added to the database.

\subsection{\label{ref:databasemenu}The Database Menu}

\begin{description}
  \opt{tc_ramcache}{
  \item[Load To RAM]
    The database can either be kept on \disk{} (to save memory), or
    loaded into RAM (for fast browsing). Setting this to \setting{Yes} loads
    the database to RAM, allowing faster browsing and searching. Setting this
    option to \setting{No} keeps the database on the \disk{}, meaning slower 
    browsing but it does not use extra RAM and saves some battery on boot up. 
    
    \opt{HAVE_DISK_STORAGE}{
    \note{If you browse your music frequently using the database, you should
      load to RAM, as this will reduce the overall battery consumption because
      the disk will not need to spin on each search.}
    }
  }
  
\item[Auto Update]
  If \setting{Auto update} is set to \setting{on}, each time the \dap{}
  boots, the database will automatically be updated.

\item[Initialize Now]
  You can force Rockbox to rescan your disk for tagged files by
  using the \setting{Initialize Now} function in the \setting{Database
    Menu}.
  \warn{\setting{Initialize Now} removes all database files (removing
    runtimedb data also) and rebuilds the database from scratch.}

\item[Update Now]
  \setting{Update now} causes the database to detect new and deleted files
    \note{Unlike the \setting{Auto Update} function, \setting{Update Now}
      will update the database regardless of whether the \setting{Directory Cache}
      is enabled. Thus, an update using \setting{Update now} may take a long
      time.
  }
  Unlike \setting{Initialize Now}, the \setting{Update Now} function
  does not remove runtime database information.
  
\item[Gather Runtime Data]
  When enabled, rockbox will record how often and how long a track is being played, 
  when it was last played and its rating. This information can be displayed in
  the WPS and is used in the database browser to, for example, show the most played, 
  unplayed and most recently played tracks.
  
\item[Export Modifications]
  This allows for the runtime data to be exported to the file \\
  \fname{/.rockbox/database\_changelog.txt}, which backs up the runtime data in
  ASCII format. This is needed when database structures change, because new
  code cannot read old database code. But, all modifications
  exported to ASCII format should be readable by all database versions.
  
\item[Import Modifications.]
  Allows the \fname{/.rockbox/database\_changelog.txt} backup to be 
  conveniently loaded into the database. If \setting{Auto Update} is
  enabled this is performed automatically when the database is initialized.
  
\end{description}

\subsection{Using the Database}
Once the database has been initialized, you can browse your music 
by Artist, Album, Genre, Song Name, etc.  To use the database, go to the
 \setting{Main Menu} and select \setting{Database}.\\

\note{You may need to increase the value of the \setting{Max Entries in File
Browser} setting (\setting{Settings $\rightarrow$ General Settings
$\rightarrow$ System $\rightarrow$ Limits}) in order to view long lists of
tracks in the ID3 database browser.\\

There is no option to turn off database completely. If you do not want
to use it just do not do the initial build of the database and do not load it
to RAM.}%

\begin{table}
  \begin{rbtabular}{.75\textwidth}{XXX}%
  {\textbf{Tag}   & \textbf{Type}  & \textbf{Origin}}{}{}
  filename              & string    & system \\ 
  album                 & string    & id tag \\
  albumartist           & string    & id tag \\
  artist                & string    & id tag \\
  comment               & string    & id tag \\
  composer              & string    & id tag \\
  genre                 & string    & id tag \\
  grouping              & string    & id tag \\
  title                 & string    & id tag \\
  bitrate               & numeric   & id tag \\
  discnum               & numeric   & id tag \\
  year                  & numeric   & id tag \\
  tracknum              & numeric   & id tag/filename \\
  autoscore             & numeric   & runtime db \\
  lastplayed            & numeric   & runtime db \\
  playcount             & numeric   & runtime db \\
  Pm (play time -- min)  & numeric   & runtime db \\
  Ps (play time -- sec)  & numeric   & runtime db \\
  rating                & numeric   & runtime db \\
  commitid              & numeric   & system \\
  entryage              & numeric   & system \\
  length                & numeric   & system \\
  Lm (track len -- min)  & numeric   & system \\
  Ls (track len -- sec)  & numeric   & system \\
  \end{rbtabular}
\end{table}

% $Id$ %
\section{\label{ref:WPS}While Playing Screen}
The While Playing Screen (WPS) displays various pieces of information about the
currently playing audio file.
%
The appearance of the WPS can be configured using WPS configuration files.
The items shown depend on your configuration -- all items can be turned on
or off independently. Refer to \reference{ref:wps_tags} for details on how
to change the display of the WPS.
\begin{itemize}
\item Status bar: The Status bar shows Battery level, charger status,
  volume, play mode, repeat mode, shuffle mode\opt{rtc}{ and clock}.
  In contrast to all other items, the status bar is always at the top of
  the screen.
\item (Scrolling) path and filename of the current song.
\item The ID3 track name.
\item The ID3 album name.
\item The ID3 artist name.
\item Bit rate. VBR files display average bitrate and ``(avg)''
\item Elapsed and total time.
\item A slidebar progress meter representing where in the song you are.
\item Peak meter.
\end{itemize}
%

See \reference{ref:ConfiguringtheWPS} for details of customising
your WPS (While Playing Screen).


\subsection{\label{ref:WPS_Key_Controls}WPS Key Controls}

  \begin{btnmap}
      \ActionWpsVolUp{} / \ActionWpsVolDown
      \opt{HAVEREMOTEKEYMAP}{& \ActionRCWpsVolUp{} / \ActionRCWpsVolDown}
      & Volume up/down.\\
      %
      \ActionWpsSkipPrev
       \opt{HAVEREMOTEKEYMAP}{& \ActionRCWpsSkipPrev}
      & Go to beginning of track, or if pressed while in the
        first seconds of a track, go to the previous track.\\
      %
      \ActionWpsSeekBack
      \opt{HAVEREMOTEKEYMAP}{& \ActionRCWpsSeekBack}
      & Rewind in track.\\
      %
      \ActionWpsSkipNext
      \opt{HAVEREMOTEKEYMAP}{& \ActionRCWpsSkipNext}
      & Go to the next track.\\
      %
      \ActionWpsSeekFwd
      \opt{HAVEREMOTEKEYMAP}{& \ActionRCWpsSeekFwd}
      & Fast forward in track.\\
      %
      \ActionWpsPlay
      \opt{HAVEREMOTEKEYMAP}{& \ActionRCWpsPlay}
      & Toggle play/pause.\\
      %
      \ActionWpsStop
      \opt{HAVEREMOTEKEYMAP}{& \ActionRCWpsStop}
      & Stop playback.\\
      %
      \ActionWpsBrowse
      \opt{HAVEREMOTEKEYMAP}{& \ActionRCWpsBrowse}
      & Return to the \setting{File Browser} / \setting{Database}.\\
      %
      \ActionWpsContext
      \opt{HAVEREMOTEKEYMAP}{& \ActionRCWpsContext}
      & Enter \setting{WPS Context Menu}.\\
      %
      \ActionWpsMenu
      \opt{HAVEREMOTEKEYMAP}{& \ActionRCWpsMenu}
      & Enter \setting{Main Menu}%
      .\\%
      %
      \opt{quickscreen}{%
        \ActionWpsQuickScreen
        \opt{HAVEREMOTEKEYMAP}{& \ActionRCWpsQuickScreen}
          & Switch to the \setting{Quick Screen}
          (see \reference{ref:QuickScreen}). \\}%
      %
      % software hold targets
      \nopt{hold_button}{%
          \opt{SANSA_CLIP_PAD}{\ButtonHome+\ButtonSelect}
          \opt{SANSA_FUZEPLUS_PAD,AIGO_EROSQ_PAD}{\ButtonPower}
          & Key lock (software hold switch) on/off.\\
      }%
      % We explicitly list all the appropriate targets here and do no condition
      % on the 'pitchscreen' feature since some players have the feature but do
      % not have the button to go from the WPS to the pitch screen.
      \opt{IRIVER_H100_PAD,IRIVER_H300_PAD,IRIVER_H10_PAD,MROBE100_PAD%
          ,GIGABEAT_PAD,GIGABEAT_S_PAD,SANSA_E200_PAD,SANSA_C200_PAD,SANSA_FUZEPLUS_PAD}{%
        \ActionWpsPitchScreen
        \opt{HAVEREMOTEKEYMAP}{& \ActionRCWpsPitchScreen}
          & Show \setting{Pitch Screen} (see \reference{sec:pitchscreen}).\\%
      }%
      \opt{GIGABEAT_PAD,GIGABEAT_S_PAD,SANSA_CLIP_PAD,MROBE100_PAD,PBELL_VIBE500_PAD%
          ,SAMSUNG_YH92X_PAD,SAMSUNG_YH820_PAD,XDUOO_X3_PAD}{%
        \ActionWpsPlaylist
        \opt{HAVEREMOTEKEYMAP}{&}
          & Show current \setting{Playlist}.\\%
      }%
      \opt{IRIVER_H100_PAD,IRIVER_H300_PAD,IRIVER_H10_PAD%
          ,SANSA_E200_PAD,SANSA_C200_PAD,SANSA_FUZEPLUS_PAD}{%
        \ActionWpsIdThreeScreen
          \opt{HAVEREMOTEKEYMAP}{& \ActionRCWpsIdThreeScreen}
          & Enter \setting{ID3 Viewer}.\\%
      }%
      \opt{hotkey}{%
        \ActionWpsHotkey \opt{HAVEREMOTEKEYMAP}{& }
        & Activate the \setting{Hotkey} function (see \reference{ref:Hotkeys}).\\
      }
      \opt{ab_repeat_buttons}{%
         \ActionWpsAbSetBNextDir{} or }%
         % not all targets have the above action defined but the one below works on all
      \nopt{erosqnative}{
        Short \ActionWpsSkipNext{} + Long \ActionWpsSkipNext
        \opt{HAVEREMOTEKEYMAP}{
          &
          \opt{IRIVER_RC_H100_PAD}{\ActionRCWpsAbSetBNextDir{} or}
          Short \ActionRCWpsSkipNext{} + Long \ActionRCWpsSkipNext}
        & Skip to the next directory.\\
        %
        \opt{ab_repeat_buttons}{%
           \ActionWpsAbSetAPrevDir{} or }%
        Short \ActionWpsSkipPrev{} + Long \ActionWpsSkipPrev
        \opt{HAVEREMOTEKEYMAP}{
          &
          \opt{IRIVER_RC_H100_PAD}{\ActionRCWpsAbSetAPrevDir{} or}
          Short \ActionRCWpsSkipPrev{} + Long \ActionRCWpsSkipPrev}
        & Skip to the previous directory.\\
      }
      %
      \opt{SANSA_E200_PAD,SANSA_C200_PAD,IRIVER_H100_PAD,IRIVER_H300_PAD}{
        \ActionStdRec
          \opt{HAVEREMOTEKEYMAP}{&}
          & Switch to the \setting{Recording Screen}.\\
      }%
  \end{btnmap}


\subsection{\label{ref:peak_meter}Peak Meter}
The peak meter can be displayed on the While Playing Screen and consists of
several indicators.
\opt{recording}{
  For a picture of the peak meter, please see the While
  Recording Screen in \reference{ref:while_recording_screen}.
}
\opt{ipodvideo}{
  \note{Especially the \playerman{} \playertype{}'s performance and battery runtime
   suffers when this feature is enabled. For this \dap{} it is highly recommended
   to not use peak meter.}
}

\begin{description}
\item [The bar:]
  This is the wide horizontal bar. It represents the current volume value.
\item [The peak indicator:]
  This is a little vertical line at the right end of the bar. It indicates
  the peak volume value that occurred recently.
\item [The clip indicator:]
  This is a little black block that is displayed at the very right of the
  scale when an overflow occurs. It usually does not show up during normal
  playback unless you play an audio file that is distorted heavily.
  \opt{recording}{
    If you encounter clipping while recording, your recording will sound distorted.
    You should lower the gain.
  }
  \note{Note that the clip detection is not very precise.
   Clipping might occur without being indicated.}
\item [The scale:]
  Between the indicators of the right and left channel there are little dots.
  These dots represent important volume values. In linear mode each dot is a
  10\% mark. In dBFS mode the dots represent the following values (from right
  to left): 0~dB, {}-3~dB, {}-6~dB, {}-9~dB, {}-12~dB, {}-18~dB, {}-24~dB, {}-30~dB,
  {}-40~dB, {}-50~dB, {}-60~dB.
\end{description}

\subsection{\label{sec:contextmenu}The WPS Context Menu}
Like the context menu for the \setting{File Browser}, the \setting{WPS Context Menu}
allows you quick access to some often used functions.

\subsubsection{Playlist}
The \setting{Playlist} submenu allows you to view, save, search, reshuffle,
and display the play time of the current playlist. These and other operations
are detailed in \reference{ref:working_with_playlists}. To change settings for
the \setting{Playlist Viewer} press \ActionStdContext{} while viewing the
current playlist to bring up the \setting{Playlist Viewer Menu}. In this
menu, you can find the \setting{Playlist Viewer Settings}.

\paragraph{Playlist Viewer Settings}
  \begin{description}
    \item[Show Icons.] This toggles display of the icon for the currently
    selected playlist entry and the icon for moving a playlist entry
    \item[Show Indices.] This toggles display of the line numbering for
       the playlist
    \item[Track Display.] This toggles between filename only and full path
       for playlist entries
  \end{description}


\subsubsection{Playlist catalogue}
  \begin{description}
    \item [Add to playlist.] Adds the currently playing file to a playlist.
    Select the playlist you want the file to be added to and it will get
    appended to that playlist.
    \item [Add to new playlist.] Similar to the previous entry this will
    add the currently playing track to a playlist. You need to enter a name
    for the new playlist first.
  \end{description}

\subsubsection{Sound Settings}
This is a shortcut to the \setting{Sound Settings Menu}, where you can configure volume,
bass, treble, and other settings affecting the sound of your music.
See \reference{ref:configure_rockbox_sound} for more information.

\subsubsection{Playback Settings}
This is a shortcut to the \setting{Playback Settings Menu}, where you can configure shuffle,
repeat, party mode, skip length and other settings affecting the playback of your music.

\subsubsection{Rating}
The menu entry is only shown if \setting{Gather Runtime Information} is
enabled. It allows the assignment of a personal rating value (0 -- 10)
to a track which can be displayed in the WPS and used in the Database
browser. The value wraps at 10.

\subsubsection{Bookmarks}
This allows you to create a bookmark in the currently-playing track.

\subsubsection{\label{ref:trackinfoviewer}Show Track Info}
\screenshot{rockbox_interface/images/ss-id3-viewer}{The track info viewer}{}
This screen is accessible from the WPS screen, and provides a detailed view of
all the identity information about the current track. This info is known as
meta data and is stored in audio file formats to keep information on artist,
album etc. To access this screen, %
\opt{IRIVER_H100_PAD,IRIVER_H300_PAD,IRIVER_H10_PAD,%
      SANSA_C200_PAD,SANSA_E200_PAD,SANSA_FUZE_PAD,SANSA_FUZEPLUS_PAD}{
  press \ActionWpsIdThreeScreen. }%
\opt{IPOD_4G_PAD,IPOD_3G_PAD,IAUDIO_X5_PAD,IAUDIO_M3_PAD,FIIO_M3K_PAD,%
      GIGABEAT_PAD,GIGABEAT_S_PAD,MROBE100_PAD,SANSA_CLIP_PAD,PBELL_VIBE500_PAD,%
      MPIO_HD200_PAD,MPIO_HD300_PAD,SAMSUNG_YH92X_PAD,SAMSUNG_YH820_PAD,XDUOO_X3_PAD}%
      {press \ActionWpsContext{} to access the
      \setting{WPS Context Menu} and select \setting{Show Track Info}. }

\subsubsection{Open With...}
This \setting{Open With} function is the same as the \setting{Open With}
function in the file browser's \setting{Context Menu}.

\subsubsection{Delete}
Delete the currently playing file. The file will be deleted but the playback
of the file will not stop immediately. Instead, the part of the file that
has already been buffered (i.e. read into the \daps\ memory) will be played.
This may even be the whole track.

\opt{pitchscreen}{
  \subsubsection{\label{sec:pitchscreen}Pitch}

  The \setting{Pitch Screen} allows you to change the rate of playback
  (i.e. the playback speed and at the same time the pitch) of your
  \dap.  The rate value can be adjusted
  between 50\% and 200\%. 50\% means half the normal playback speed and a
  pitch that is an octave lower than the normal pitch. 200\% means double
  playback speed and a pitch that is an octave higher than the normal pitch.

  The rate can be changed in two modes: procentual and semitone.
  Initially, procentual mode is active.

    If you've enabled the \setting{Timestretch} option in
    \setting{Sound Settings} and have since rebooted, you can also use
    timestretch mode. This allows you to change the playback speed
    without affecting the pitch, and vice versa.

    In timestretch mode there are separate displays for pitch and
    speed, and each can be altered independently.  Due to the
    limitations of the algorithm, speed is limited to be between 35\%
    and 250\% of the current pitch value.  Pitch must maintain the
    same ratio as well as remain between 50\% and 200\%.

  The value of the rate, pitch and speed
  is not persistent, i.e. after the \dap\ is turned on it will
  always be set to 100\%.  However, the rate, pitch and speed
  information will be stored in any bookmarks you may create
  (see \reference{ref:Bookmarkconfigactual}) and will be restored upon
  playing back those bookmarks.

  \begin{btnmap}
    \ActionPsToggleMode
    \opt{HAVEREMOTEKEYMAP}{& \ActionRCPsToggleMode}
    & Toggle pitch changing mode (cycle through all available modes).\\
    %
    \ActionPsIncSmall{} / \ActionPsDecSmall
    \opt{HAVEREMOTEKEYMAP}{& \ActionRCPsIncSmall{} / \ActionRCPsDecSmall}
    & Increase~/ Decrease pitch by 0.1\% (in procentual mode) or 0.1
      semitone (in semitone mode).\\
    %
    \nopt{PBELL_VIBE500_PAD}{ % there is no long scroll up or down because of slide
    \ActionPsIncBig{} / \ActionPsDecBig
    \opt{HAVEREMOTEKEYMAP}{& \ActionRCPsIncBig{} / \ActionRCPsDecBig}
    & Increase~/ Decrease pitch by 1\% (in procentual mode) or a semitone
      (in semitone mode).\\
    }
    %
    \ActionPsNudgeLeft{} / \ActionPsNudgeRight
    \opt{HAVEREMOTEKEYMAP}{& \ActionRCPsNudgeLeft{} / \ActionRCPsNudgeRight}
    & Temporarily change pitch by 2\% (beatmatch), or modify speed (in timestretch mode).\\
    %
    \ActionPsReset
    \opt{HAVEREMOTEKEYMAP}{& \ActionRCPsReset}
    & Reset pitch and speed to 100\%. \\
    %
    \ActionPsExit
    \opt{HAVEREMOTEKEYMAP}{& \ActionRCPsExit}
    & Leave the \setting{Pitch Screen}. \\
    %
  \end{btnmap}

}


%Include playlist section
% $Id$ %
\chapter{Introduction}
\section{Welcome}
This is the manual for Rockbox. Rockbox is an open source firmware replacement
for a growing number of digital audio players. Rockbox aims to be considerably
more functional and efficient than your device's stock firmware while remaining
easy to use and customisable. Rockbox is written by users, for users. Not only
is it free to use, it is also released under the GNU General Public License
(GPL), which means that it will always remain free both to use and to change.

Rockbox has been in development since 2001, and receives new features, tweaks
and fixes each day to provide you with the best possible experience on your
digital audio player. A major goal of Rockbox is to be simple and easy to use,
yet remain very customisable and configurable. We believe that you should never
need to go through a series of menus for an action you perform frequently. We
also believe that you should be able to configure almost anything about Rockbox
you could want, pertaining to functionality. Another top priority of Rockbox is
audio playback quality -- Rockbox, for most models, includes a wider range of
sound settings than the device's original firmware. A lot of work has been put
into making Rockbox sound the best it can, and improvements are constantly being
made. All models have access to a large number of plugins, including many games,
applications, and graphical ``demos''. You can load different configurations
quickly for different purposes (e.g. a large font for in your car, different
sound settings for at home). Rockbox features a very wide range of languages, and
all supported models also have the ability to talk to you -- menus can be voiced
and filenames spelled out or spoken.

\section{Getting more help}
This manual is intended to be a comprehensive introduction to the Rockbox
firmware. There is, however, more help available. The Rockbox website at
\url{https://www.rockbox.org/} contains very extensive documentation and guides
written by members of the Rockbox community and this should be your first port
of call when looking for further help.

If you cannot find the information you are searching for on the Rockbox
website there are a number of support channels you should have a look at.
You can try the Rockbox forums located at \url{https://forums.rockbox.org/}.
The mailing lists are another option, and can be found at
\url{https://www.rockbox.org/mail/}. From that page you can subscribe to the
lists and browse the archives. To search the list archives simply use
the search field that is located on the left side of the website.
Furthermore,  you can ask on IRC. The main channel for Rockbox is
\texttt{\#rockbox} on \url{irc://irc.libera.chat}. Many helpful developers
and users are usually around. Just join and ask your question (don't ask to
ask!) -- if someone knows the answer you'll
usually get an answer pretty quickly. More information including IRC logs
can be found at \url{https://www.rockbox.org/irc/}. We also have a web client
so that you can join the Rockbox IRC channel without needing
to install additional software onto your computer.

If you think you have found a bug please make sure it actually is a bug and is
still present in the most recent version of Rockbox. You should try to
confirm that by using the above mentioned support channels first. After that
you can submit that issue to our tracker. Refer to \reference{sec:feedback}
for details on how to use the tracker.


\section{Naming conventions and marks}
We have some conventions (especially for naming) that are intended to be
consistent throughout this manual.

Manufacturer and product names are formatted in accordance with the standard
rules of English grammar, e.g. ``\playerman{} playback is currently
unsupported''. Manufacturer and model names are proper nouns, and
thus are written beginning with a capital letter.

% write a bit more about names etc. here.
\Ifpdfoutput{
This manual has some parts that are marked with icons on the margin to help
you finding important parts or parts you could skip. The following icons
are used:
\\
\note{This indicates a note. A note starts always with the text ``Note''.
  In order to make finding notes easier each one is accompanied by
  an icon in the margin as here. Notes are used to mark useful information
  that may help you to get the most out of Rockbox.
\\
}
\warn{This is a warning. In contrast to notes mentioned above, a warning
  should be taken more seriously. Whereas ignoring notes will not cause any
  serious damage, ignoring warnings \emph{could} cause serious damage to 
  your \dap{}. You really should read the warnings, especially if you are
  new to Rockbox.
\\
}
\blind{This icon marks a section that is intended especially for the blind
  and visually impaired. As they cannot
  read the manual in the same way sighted people do we have added some
  additional descriptions. If you are not blind or visually impaired you
  can probably completely skip these blocks. To make this easier, there is an
  icon shown in the margin on the right.
\\
}
}{}% end Ifpdfoutput

Links to the wiki are abbreviated by the name of the wiki page. Those names
are still linked so you can simply follow them like any other link in this
manual. If you want to access a wiki page manually go to
\wikiicon{} \href{\wikibaseurl}{\wikibaseurl}
and type the page name in the ``Go'' box at the top of the page.
\Ifpdfoutput{Links to wiki pages are also indicated by the symbol \wikiicon{}
in front of the page name.}{}

\chapter{Installation}\label{sec:installation}

Installing Rockbox is generally a quick and easy procedure. However
before beginning there are a few important things to know.

\section{Before Starting}

\opt{e200}{\fixme{NOTE: These instructions will not work on the
``Rhapsody'' version of the E200 series (also known as E200R). Please
follow the instructions at
\wikilink{SansaE200RInstallation}.}}

\opt{ipodnano,ipodnano2g,ipodvideo,ipod6g,e200,c200,c200v2,e200v2,clipv1,clipv2,cowond2,fuze,fuzev2}{
\begin{description}
\item[Supported hardware versions.]
    \opt{ipodnano,ipodnano2g}{
    The \playertype{} is available in multiple versions, not
    all of which run Rockbox.  Rockbox presently runs only on
    the first and second generation Ipod Nano. Rockbox does
    \emph{not} run on the third, fourth or fifth generation Ipod Nano.
    For information on identifying which Ipod you own, see this page on
    Apple's website: \url{http://www.info.apple.com/kbnum/n61688}.
  }
  \opt{ipodvideo}{
    The \playertype{} is the 5th/5.5th generation \playerman{} only.
    For information on identifying which Ipod you own, see this page on Apple's
    website: \url{http://www.info.apple.com/kbnum/n61688}.
  }
  \opt{ipod6g}{
    The \playertype{} refers to the 6th generation model of the
    \playerman{}. It comes with disk sizes of 80GB, 120GB, and 160GB
    in ``thick'' and ``slim'' versions.
  }
  \opt{c200,c200v2,e200,e200v2}{
    The \playertype{} is available in multiple versions, and you need to make
    sure which you have by checking the Sandisk firmware version number under
    Settings $\rightarrow$ Info. The v1 firmware is named 01.xx.xx, while the
    v2 firmware begins with 03. Make sure that you are following the
    instructions from the correct manual.
}
  \opt{clipv1,clipv2,fuze,fuzev2}{
    The \playertype{} is available in multiple versions, and you need to make
    sure which you have by checking the Sandisk firmware version number under
    Settings $\rightarrow$ \opt{fuze,fuzev2}{System Settings $\rightarrow$}
    Info. The v1 firmware is named 01.xx.xx, while the v2 firmware begins with
    02. Make sure that you are following the instructions from the correct
    manual.
}
  \opt{cowond2}{
    Rockbox runs on all \playerman{} \playertype{} and \playertype{}+ variants
    (2 / 4 / 8 / 16~GB, with or without DAB/DMB).
    \note{Newer \playertype{}+ hardware revisions use an updated power
    management chip, and some functionality is not yet implemented on these
    players (e.g. touchscreen support).}
}
\end{description}
}

\opt{cowond2}{
\begin{description}
  \item[Current limitations.] Most Rockbox functions are usable on the
    \playertype{}/\playertype{}+, including music playback and most plugins, but
    there are a number of shortcomings that prevent it from being a fully
    supported target:
    \begin{itemize}
      \item An SD card is required to use many features, since the internal
        flash memory is read-only in Rockbox.
        \warn{The SD driver is still in development and may contain bugs.
          There have been reports of SD cards becoming unusable after being used
          with Rockbox on \playerman{} \playertype{}. Only use old, low-capacity
          cards until we are satisfied the driver is safe to use.\\}
      \item There is only basic touchscreen support. Further work is
        required to make the UI more usable with a touchscreen in general.
        \note{The touchscreen can be used in two modes, either a 3$\times$3 grid mode
          which divides the screen into areas to emulate a set of physical
          buttons (the default setting), or ``absolute point'' mode where the
          touchscreen is used to point to items on the screen.\\}
    \end{itemize}
\end{description}
}
\opt{ondavx777}{
\begin{description}
  \item[Current limitations.] Most Rockbox functions are usable on the
    \playertype{}, including music playback and most plugins, but
    there are a number of shortcomings that prevent it from being a fully
    supported target:
    \begin{itemize}
      \item A MicroSD card is required to run Rockbox, since the internal
        flash memory hasn't been figured out yet.
      \item There is only basic touchscreen support. Further work is
        required to make the UI more usable with a touchscreen in general.
        \note{The touchscreen can be used in two modes, either a 3$\times$3 grid mode
          which divides the screen into areas to emulate a set of physical
          buttons (the default setting), or ``absolute point'' mode where the
          touchscreen is used to point to items on the screen.\\}
    \end{itemize}
\end{description}
}
\opt{iriverh300}{
\begin{description}
  \item[DRM capability.] If your \dap{} has a US firmware, then by installing Rockbox you will
  \emph{permanently} lose the ability to playback files with DRM.
\end{description}
}

\opt{sansaAMS,fuzeplus}{
\begin{description}
  \item[DRM capability.] It is possible that installation of the bootloader
  may lead to you \emph{permanently} losing the ability to playback files
  with DRM.
\end{description}
}

\opt{fiiom3k,shanlingq1,agptekrocker,xduoox3ii,xduoox20,aigoerosq,erosqnative}{
Although Rockbox is considered fully functional on the \playername{}, there are
a few limitations compared to the original firmware which you should be aware
of before installing.
\begin{description}
  \item[Filesystem support.] Rockbox only supports the FAT32 filesystem. Other
    filesystems such as exFAT or NTFS are not supported.
  \item[USB DAC.] This feature is not supported by Rockbox, but you can
    dual-boot the original firmware if you want to use it.
  \opt{shanlingq1,agptekrocker,xduoox3ii,xduoox20,aigoerosq,erosqnative}{\item[Wireless.] There is no support for Bluetooth\opt{shanlingq1}{ or WiFi}.
    You can dual-boot the original firmware to use wireless functionality.}
  \item[Hotswapping SD cards.] Rockbox runs from the SD card, not the internal
    flash memory. Hotswapping is technically possible, but you need to install
    the same version of Rockbox to each SD card.

    Some features might not work correctly after hotswapping and you may
    experience crashes or instability. Removing the SD card while data is being
    written is liable to cause crashes and data loss, and possibly corrupt your
    filesystem.
  \opt{shanlingq1}{\item[Touchscreen.] There is only basic touchscreen support.
    Further work is required to make the UI more usable with a touchscreen in
    general.
    \note{The touchscreen can be used in two modes, either a 3$\times$3 grid mode
      which divides the screen into areas to emulate a set of physical
      buttons (the default setting), or ``absolute point'' mode where the
      touchscreen is used to point to items on the screen.\\}}
\end{description}
}

\nopt{gigabeats,fiiom3k,shanlingq1}{
\begin{description}

\nopt{ipod1g2g}{
  \item[USB connection.]
}
\opt{ipod1g2g}{
  \item[Firewire connection.]
}
  To transfer Rockbox to your \dap{} you need to
  connect it to your computer. For manual installation/uninstallation, or
  should autodetection fail during automatic installation, you need to know
  where to access the \dap{}. On Windows this means you need to know
  the drive letter associated with the \dap{}. On Linux you need to know
  the mount point of your \dap{}. On Mac OS X you need to know the volume
  name of your \dap{}.

  \opt{ipod}{
    If you have Itunes installed and it is configured to open automatically
    when your \dap{} is attached (the default behaviour), then wait for it to
    open and then quit it. You also need to ensure the ``Enable use as disk''
    option is enabled for your \dap{} in Itunes. Your \dap{} should then enter
    disk mode automatically when connected to a computer via
    \nopt{ipod1g2g}{USB.}\opt{ipod1g2g}{Firewire.} If your computer does not
    recognise your \dap{}, you may need to enter disk mode manually. Disconnect
    your \dap{} from the computer. Hard reset the \dap{} by pressing and
    holding the \ButtonMenu{} and \nopt{IPOD_3G_PAD}{\ButtonSelect{}}%
    \opt{IPOD_3G_PAD}{\ButtonPlay{}} buttons simultaneously. As soon as the
    \dap{} resets, press and hold the \nopt{IPOD_3G_PAD}{\ButtonSelect{} and
    \ButtonPlay{}}\opt{IPOD_3G_PAD}{\ButtonLeft{} and \ButtonRight{}} buttons
    simultaneously. Your \dap{} should enter disk mode and you can try
    reconnecting to the computer.
  }
  \opt{iaudiox5}{
    When instructed to connect/disconnect the USB cable, always use
    the USB port through the subpack, not the side `USB Host' port. The side port
    is intended to be used for USB OTG connections only (digital cameras, memory
    sticks, etc.).
  }
  \opt{sansa,e200v2,clipv1,clipv2,fuzeplus,c200v2}{
    \note{The following steps require you to change the setting in
    \setting{Settings $\rightarrow$ USB Mode} to \setting{MSC} from within the
    original firmware.}

    \nopt{sansaAMS,fuzeplus}{
        \warn{Never extract files to your \dap{} while it
        is in recovery mode.}
    }
  }

  \opt{fuze,fuzev2,clipplus,clipzip}{
    \note{The following steps require you to change the setting in
    \setting{Settings $\rightarrow$ System Settings $\rightarrow$ USB Mode} to
    \setting{MSC} from within the original firmware.  Further note that
     all original firmware settings will be lost immediately after patching
     the bootloader, so you may need to correct this setting again after
     installing rockbox.}
  }

  \opt{iriverh10,iriverh10_5gb}{
    The installation requires you to use UMS mode and so
    may require use of the UMS trick, whereby it is possible to force a MTP
    \playertype{} to start up in UMS mode as follows:
      \begin{enumerate}
        \item Ensure the \dap{} is fully powered off by \opt{iriverh10}{using
          a pin to push the small reset button inside the hole between the
          \ButtonHold{} switch and remote control connector.}\opt{iriverh10_5gb}{removing
          the battery and putting it back in again.}
        \item Connect your \playertype{} to the computer using the data cable.
        \item Hold \ButtonRight{} and push \ButtonPower{} to turn the \dap{} on.
        \item Continue holding \ButtonRight{} until the USB Connected screen appears.
        \item The \dap{} will now appear as a regular disk on your computer.
      \end{enumerate}
      \note{Once Rockbox has been installed, when you shut down your \dap{} from Rockbox it will totally
       power the player off so step 1 is no longer necessary.}
  }
  \opt{gigabeatf}{The installation requires you to change a setting in the
      original firmware.  Make sure the option under \setting{Setup
      $\rightarrow$ Connections $\rightarrow$ PC Connections} is set to
      \setting{gigabeat room}. Also, during installation, do not connect your
      \dap{} using the cradle but plug the USB cable directly to the \dap{}.
  }
  \opt{vibe500,samsungyh}{
    The installation requires you to use UMS mode.
    In order to start up your \playertype{} in UMS mode you need to:
      \begin{enumerate}
        \item Turn off the \dap{} (Original Firmware).
        \item Connect your \playertype{} to the computer using the data cable.
        \opt{vibe500}{
        \item Push and hold \ButtonPower{} (for about 2 seconds) until the ``USB'' screen appears.
        }
      \end{enumerate}
    The \dap{} will now appear as a regular disk on your computer.
  }
\end{description}
}

\opt{ipod,sansa}{
\begin{description}
  \item[Administrator/Root rights.] Installing the bootloader portion of Rockbox
  requires you to have administrative (Windows) or root (Linux) rights.
  Consequently when doing either the automatic or manual bootloader install,
  please ensure that you are logged in with an administrator account or have root rights.
\end{description}
}

\opt{ipod}{
\begin{description}
  \item[File system format.] Rockbox only works on Ipods formatted with
  the FAT32 filesystem (i.e. Ipods initialised by Itunes
  for Windows). It does not work with the HFS+ filesystem (i.e. Ipods
  initialised by Itunes for the Mac). More information and instructions for
  converting an Ipod to FAT32 can be found on the
  \wikilink{IpodConversionToFAT32} wiki
  page on the Rockbox website. Note that after conversion, you can still use
  a FAT32 Ipod with a Mac.
\end{description}
}

\section{Installing Rockbox}\label{sec:installing_rockbox}\index{Installation}
There are two ways to install Rockbox: automated and manual. The automated
way is the preferred method of installing Rockbox for the majority of
people. Rockbox Utility is a graphical application that does almost everything
for you. However, should you encounter a problem, then the manual way is
still available to you.\\

\opt{gigabeats,fiiom3k,shanlingq1,erosqnative}{
  \note{The automated install is not yet available for the
  \playerlongtype{}. For now you can use the manual method to install Rockbox.
  Please still read the section on the automatic install as it explains
  various important aspects of Rockbox, such as the different versions
  available.\\}}

  \opt{HAVE_RB_BL_ON_DISK}{There are three separate components,
    two of which need to be installed in order to run Rockbox:}
  \opt{HAVE_RB_BL_IN_FLASH}{There are two separate components
    which need to be installed in order to run Rockbox:}

\begin{description}
\opt{HAVE_RB_BL_ON_DISK}{
\item[The \playerman{} bootloader.]
  The \playerman{} bootloader is the program that tells your \dap{} how to load
  and start the original firmware. It is also responsible for any emergency,
  recovery, or disk modes on your \dap{}. This bootloader is stored in special flash
  memory in your \playerman{} and comes factory-installed. It is not necessary
  to modify this in order to install Rockbox.}

\item[The Rockbox bootloader.] \index{Bootloader}
  \opt{HAVE_RB_BL_ON_DISK}{The Rockbox bootloader is loaded from disk by
  the \playerman{} bootloader. It is responsible for loading the Rockbox
  firmware and for providing the dual boot function. It directly replaces the
  \playerman{} firmware in the \daps{} boot sequence.
  \opt{gigabeatf}{\note{Dual boot does not currently work on the Gigabeat.}}}

  \opt{HAVE_RB_BL_IN_FLASH}{
  The bootloader is the program that tells your
  \dap{} how to load and start other components of Rockbox and for providing
  the dual boot function. This is the component of Rockbox that is installed
  to the flash memory of your \playerman.
  \opt{iaudiom3,iaudiom5,iaudiox5}{\note{Dual boot does not currently work on the \playertype.}}}

\item[The Rockbox firmware.]
  \opt{HAVE_RB_BL_IN_FLASH}{Unlike the \playerman{} firmware, which runs
  entirely from flash memory,}
  \opt{HAVE_RB_BL_ON_DISK}{Similar to the \playerman{} firmware,}
  most of the Rockbox code is contained in a
  ``build'' that resides on your \daps{} drive. This makes it easy to
  update Rockbox. The build consists of a directory called
  \fname{.rockbox} which contains all of the Rockbox files, and is
  located in the root of your \daps{} drive.
\end{description}

Apart from the required parts there are some addons you might be interested
in installing.
\begin{description}
\item[Fonts.] Rockbox can load custom fonts. The fonts are
    distributed as a separate package and thus need to be installed
    separately. They are not required to run Rockbox itself but
    a lot of themes require the fonts package to be installed.

\item[Themes.] The appearance of Rockbox can be customised by themes. Depending
    on your taste you might want to install additional themes to change
    the look of Rockbox.
\end{description}

\subsection{Automated Installation}

To automatically install Rockbox, download the official installer and
housekeeping tool \caps{Rockbox Utility}. It allows you to:
\begin{itemize}
\item Automatically install all needed components for using Rockbox
        (``Minimal Installation'').
\item Automatically install all suggested components (``Complete Installation'').
\item Selectively install optional components.
\item Install additional fonts and themes.
\item Install voice files and generate talk clips.
\item Uninstall all components you installed using Rockbox Utility.
\end{itemize}

Prebuilt binaries for Windows, Linux and Mac OS X are
available at the \wikilink{RockboxUtility} wiki page.\\

\opt{gigabeats,ondavx777,fiiom3k,shanlingq1,erosqnative}{
\note{Rockbox Utility does not currently support the \playertype{} and you will
therefore need to follow the manual install instructions below.\\}}

When first starting \caps{Rockbox Utility} run ``Autodetect'',
found in the configuration dialog (File $\rightarrow$ Configure). Autodetection
can detect most player types. If autodetection fails or is unable to detect
the mountpoint, make sure to enter the correct values. The mountpoint indicates
the location of the \dap{} in your filesystem. On Windows, this is the drive
letter the \dap{} gets assigned, on other systems this is a path in the
filesystem.\\*


\opt{iriverh100,iriverh300}{
  Rockbox Utility will ask you for a compatible copy of the original
  firmware. This is because for legal reasons we cannot distribute
  the bootloader directly. Instead, we have to patch the Iriver firmware
  with the Rockbox bootloader.

  Download a supported version of the Iriver firmware for your
  \playername{} from the Iriver website, links can be found on
  \wikilink{IriverBoot}.

  Supported Iriver firmware versions currently include
  \opt{iriverh100}{1.63US, 1.63EU, 1.63K, 1.65US, 1.65EU, 1.65K, 1.66US,
    1.66EU and 1.66K. Note that the H140 uses the same firmware as the H120;
    H120 and H140 owners should use the firmware called \fname{ihp\_120.hex}.
    Likewise, the iHP110 and iHP115 use the same firmware, called
    \fname{ihp\_100.hex}. Be sure to use the correct firmware file for
    your player.}
    \opt{iriverh300}{1.28K, 1.28EU, 1.28J, 1.29K, 1.29J, 1.30EU and 1.31K.
    \note{The US \playername{} firmware is not supported and cannot be
    patched to be used with the bootloader. If you wish to install Rockbox
    on a US \playername{}, you must first install a non-US version of the
    original firmware and then install one of the supported versions patched
    with the Rockbox bootloader.}
    \warn{Installing a non-US firmware on a US \playername{} will
    \emph{permanently} remove DRM support from the player.}}

  If the file that you downloaded is a \fname{.zip} file, use an unzip
  utility like mentioned in the prerequisites section to extract
  the \fname{.hex} from the \fname{.zip} file
  to your desktop. Likewise, if the file that you downloaded is an
  \fname{.exe} file, double-click on the \fname{.exe} file to extract
  the \fname{.hex} file to your desktop.
  When running Linux you should be able to extract \fname{.exe}
  files using \fname{unzip}.
}

\opt{mpiohd200,mpiohd300}{
  Rockbox Utility will ask you for a compatible copy of the original
  firmware. This is because for legal reasons we cannot distribute
  the bootloader directly. Instead, we have to patch the MPIO firmware
  with the Rockbox bootloader.

  Download a supported version of the MPIO firmware for your
  \playername{} from the MPIO website, links can be found on
  \opt{mpiohd200}{\wikilink{MPIOHD200Port}}%
  \opt{mpiohd300}{\wikilink{MPIOHD300Port}}.

  \warn{The only tested version of the original firmware is
  \opt{mpiohd200}{1.30.05}\opt{mpiohd300}{1.30.06}
  and as such is the only supported version}

  If the file that you downloaded is a \fname{.zip} file, use an unzip
  utility like mentioned in the prerequisites section to extract
  the \fname{.SYS} from the \fname{.zip} file
  to your desktop. Likewise, if the file that you downloaded is an
  \fname{.exe} file, double-click on the \fname{.exe} file to extract
  the \fname{.SYS} file to your desktop.
  When running Linux you should be able to extract \fname{.exe}
  files using \fname{unzip}.
}

\opt{fuzeplus}{
  Rockbox Utility will ask you for a compatible copy of the original
  firmware. This is because for legal reasons we cannot distribute
  the bootloader directly. Instead, we have to patch the \playerman{}
  firmware with the Rockbox bootloader.

  Download a supported version of the \playerman{} firmware for your
  \playername{} from the \playerman{} website, links can be found on
  \wikilink{SansaFuzePlusPort}.
  \note{Although the only tested version of the original firmware is 02.38.6,
    Rockbox Utility should be able to patch any newer version.}
  \note{If the file that you downloaded is a \fname{.zip} file, use an unzip
    utility like mentioned in the prerequisites section to extract the
    \fname{firmware.sb} from the \fname{.zip} file to your desktop.}
}
\subsubsection{Choosing a Rockbox version}\label{sec:choosing_version}

There are three different versions of Rockbox available from the
Rockbox website:
\label{Version}
Release version, current build and archived daily build. You need to decide which one
you want to install and get the appropriate version for your \dap{}. If you
select either ``Minimal Installation'' or ``Complete Installation'' from the
``Quick Start'' tab, then Rockbox Utility will automatically install the
release version of Rockbox. Using the ``Installation'' tab will allow you
to select which version you wish to install.

\begin{description}

\item[Release.] The release version is the latest stable release, free
   of known critical bugs. For a manual install, the current stable release of Rockbox is
   available at \url{https://www.rockbox.org/download/}.

\item[Development Build.] The development build is built at each change to
  the Rockbox source code repository and represents the current state of Rockbox
  development. This means that the build could contain bugs but most of
  the time is safe to use. For a manual install, you can download the current build from
  \url{https://build.rockbox.org/}.

\item[Archived Build.] In addition to the release version and the current build,
  there is also an archive of daily builds available for download. These are
  built once a day from the latest source code in the repository. For a manual install,
  you can download archived builds from \url{https://www.rockbox.org/daily.shtml}.

\end{description}

\note{Because current and archived builds are development versions that
      change frequently, they may behave differently than described in this manual,
      or they may introduce new (and potentially annoying) bugs. Unless you wish to
      try the latest and greatest features at the price of possibly greater instability,
      or you wish to help with development, you should stick with the release.\\*}

Please now go to \reference{ref:finish_install} to complete the installation procedure.

\subsection{Manual Installation}

The manual installation method is still available to you, should you need or desire it
by following the instructions below. If you have used Rockbox Utility
to install Rockbox, then you do not need to follow the next section and can skip
straight to \reference{ref:finish_install}

\opt{gigabeats}{\subsubsection{Installing the bootloader}
    \input{getting_started/gigabeats_install.tex}
}

\subsubsection{Installing the firmware}\label{sec:installing_firmware}

\opt{gigabeats}{\note{When your \dap{} is in the Rockbox USB or bootloader
USB mode, you will see two visible partitions -- the 150~MB firmware
partition (containing at least a file called \fname{nk.bin}) and
the main data partition. Rockbox \emph{must} be installed onto the main
data partition.}}

\begin{enumerate}
\item Download your chosen version of Rockbox from the links in the
  previous section.

\item Connect your \dap{} to the computer via USB
  \opt{sansa,sansaAMS,iriverh10,iriverh10_5gb,vibe500,fuzeplus,samsungyh}
  { in MSC mode }
  \opt{ipod3g,ipod4g,ipodmini,ipodcolor}{ or Firewire }as described in
  the manual that came with your \dap{}.

\item Take the \fname{.zip} file that you downloaded and use
 the ``Extract all'' command of your unzip program to extract
 the files onto
 \opt{gigabeats,fuzeplus}{the main data partition of }
 \opt{cowond2}{either an SD card or the internal memory of }
 \opt{ondavx777}{the MicroSD of }
 your \dap{}.
 \opt{cowond2}{\note{If you have chosen to extract to the internal memory, it
     will not be possible to save settings.}}
\end{enumerate}

\note{The entire contents of the \fname{.zip} file should be extracted
directly to the root of your \daps{} drive. Do not try to
create a separate directory on your \dap{} for the Rockbox
files! The \fname{.zip} file already contains the internal
structure that Rockbox needs.\\}

% This has nothing to do with swcodec, just that these players need our own
% bootloader so we can decide where we want the main binary.
If the contents of the \fname{.zip} file are extracted correctly, you will
have a directory called \fname{.rockbox}, which contains all the files needed
by Rockbox, in the main directory of your \daps{} drive.

\opt{swcodec}{%
    \nopt{gigabeats}{%
    \subsubsection{Installing the bootloader}
        \opt{iriverh100,iriverh300}{\input{getting_started/iriver_install.tex}}
        \opt{mpiohd200,mpiohd300}{\input{getting_started/mpio_install.tex}}
        \opt{ipod}{
          \nopt{ipod6g} {
            \input{getting_started/ipod_install.tex}}
          \opt{ipod6g} {
            \input{getting_started/ipod6g_install.tex}} }
        \opt{iaudio}{\input{getting_started/iaudio_install.tex}}
        \opt{iriverh10,iriverh10_5gb}{\input{getting_started/h10_install.tex}}
        \opt{gigabeatf}{\input{getting_started/gigabeat_install.tex}}
        \opt{sansa}{\input{getting_started/sansa_install.tex}}
        \opt{sansaAMS}{\input{getting_started/sansaAMS_install.tex}}
        \opt{mrobe100}{\input{getting_started/mrobe100_install.tex}}
        \opt{cowond2}{\input{getting_started/cowond2_install.tex}}
        \opt{vibe500}{\input{getting_started/vibe500_install.tex}}
        \opt{ondavx777}{\input{getting_started/ondavx777_install.tex}}
        \opt{IMX233}{\input{getting_started/IMX233_install.tex}}
        \opt{samsungyh}{\input{getting_started/samsungyh_install.tex}}
        \opt{xduoox3}{\input{getting_started/xduoox3_install.tex}}
        \opt{xduoox3ii,xduoox20,agptekrocker,aigoerosq}{\input{getting_started/hibyos_install.tex}}
        \opt{fiiom3k,shanlingq1,erosqnative}{\input{getting_started/jztool_install.tex}}
    }
}

\subsection{Finishing the install}\label{ref:finish_install}

\opt{gigabeatf}{
  After installing you \emph{need} to power-cycle the
  \dap{} by doing the following steps. Failure to do so may result in problems.
  \begin{itemize}
  \item Safely eject / unmount your \dap{} and unplug the USB cable.
  \item Unplug any power adapter.
  \item Hold the \ButtonPower{} button to turn off the \dap{}.
  \item Slide the battery switch located on the bottom of the \dap{} from
  `on' to `off'.
  \item Slide the battery switch back from `off' to `on'.
  \end{itemize}
}

\opt{iaudiom3,iaudiom5,iaudiox5}{
  After installing you \emph{need} to power-cycle the
  \dap{} by doing the following steps.
  \begin{itemize}
  \item Safely eject / unmount your \dap{} and unplug the USB cable.
  \item Hold the
     \opt{IAUDIO_X5_PAD}{\ButtonPower}
     \opt{IAUDIO_M3_PAD}{\ButtonPlay}
     button to turn off the \dap{}.
  \item Insert the charger. The Rockbox bootloader will automatically be flashed.
  \end{itemize}
}

\opt{iriverh10,iriverh10_5gb,ipod,mrobe100,sansa,cowond2,vibe500,ondavx777,samsungyh}{
  Safely eject / unmount the USB drive, unplug the cable and restart.
}

\opt{sansaAMS}{
  Safely eject the device, unplug USB and wait for the firmware update to finish.
  Don't try to power off the device, it will shutdown by itself after a minute.
}

\opt{gigabeats}{
  Safely eject / unmount your \dap{}.
}

\opt{iriverh100,iriverh300}{
  \begin{itemize}
  \item Safely eject / unmount your \dap{}.

  \item \warn{Before proceeding further, make sure that your player has a full charge
  or that it is connected to the power adapter. Interrupting the next step
  due to a power failure most likely will brick your \dap{}.}
  Update your \daps{} firmware with the patched bootloader. To do this, turn
  the jukebox on. Press and hold the \ButtonSelect{} button to enter the main menu,
  and navigate to \setting{General $\rightarrow$ Firmware Upgrade}. Select
  \setting{Yes} when asked to confirm if you want to upgrade the
  firmware. The \playerman{} will display a message indicating that the
  firmware update is in progress. Do \emph{not} interrupt this process. When the
  firmware update is complete the player will turn itself off. (The update
  firmware process usually takes a minute or so.). You are now ready to go.
\end{itemize}
}
\opt{fuzeplus}{
  \warn{Before proceeding further, make sure that your player's battery is enough charged.
    Interrupting the next step due to a power failure most likely
    will brick your \dap{}.
  }
  \note{If you are updating/reinstalling the bootloader on a previously
    rockbox installed bootloader you will need to boot into the original
    firmware in order to perform the following step. See \reference{ref:Dualboot}
    for details on how to do so
  }
  Update your \daps{} firmware with the patched bootloader. To do this,
  safely eject /unmount your \dap{}. The update process should start
  immediatly. The \playerman{} will display an animation indicating that the
  firmware update is in progress with words: ``Updating Fuze+''.
  Do \emph{not} interrupt this process. When the firmware update is complete the
  player will restart (The update firmware process usually takes one to several
  minutes.). You are now ready to go.
}
\opt{mpiohd200}{
  \begin{itemize}
  \item Safely eject /unmount your \dap{}.

  \item \warn{Before proceeding further, make sure that your player has a full charge
  or that it is connected to the power adapter. Interrupting the next step
  due to a power failure most likely will brick your \dap{}.}
  Update your \daps{} firmware with the patched bootloader. To do this, turn
  the jukebox on. The update process should start automatically. The \playerman{} will
  display animation indicating that the firmware update is in progress. Do \emph{not}
  interrupt this process. When the firmware update is complete the player will restart.
  (The update firmware process usually takes a minute or so.). You are now ready to go.
\end{itemize}
}
\opt{e200}{Your e200 will automatically reboot and Rockbox should load.}

\opt{fiiom3k,shanlingq1,erosqnative}{
  Safely eject / unmount your \dap{}.
}


\subsection{Enabling Speech Support (optional)}\label{sec:enabling_speech_support}
\index{Speech}\index{Installation!Optional Steps}
If you wish to use speech support you will also need a voice file. Voice files
allow Rockbox to speak the user interface to you. Rockbox Utility can install
an English voice file, or you can download it from \url{https://www.rockbox.org/daily.shtml}
and unzip it to the root of your \dap{}.
Rockbox Utility can also aid you in the creation of voice files with different voices
or in other languages if you have a suitable speech engine installed on your computer.
Voice menus are enabled by default and will come
into effect after a reboot. See \reference{ref:Voiceconfiguration} for details
on voice settings.
Rockbox Utility can also aid in the production of talk files, which allow Rockbox
to speak file and folder names.

\section{Running Rockbox}
\nopt{ipod,gigabeats,cowond2}{When
you turn the unit on, Rockbox should load.}
\opt{ipod}{Hard reset the Ipod by holding
  \opt{IPOD_4G_PAD}{\ButtonMenu{} and \ButtonSelect{} simultaneously}%
  \opt{IPOD_3G_PAD}{\ButtonMenu{} and \ButtonPlay{} simultaneously}
  for a couple of seconds until the \dap{} resets. Now Rockbox should load.
}

\opt{gigabeats}{Rockbox should automatically load when you turn on your player.\\

  \note{
    If you have loaded music onto your \dap{} using the \playerman{}
    firmware, you will not be able to see your music properly in the
    \setting{File Browser} as MTP mode changes the location and file names.
    Files placed on your \dap{} using the \playerman{} firmware can be
    viewed by initialising and using Rockbox's database.
    See \reference{ref:database} for more information.}
}

\opt{cowond2}{
  To boot the Rockbox firmware set the \ButtonHold{} switch immediately after
  power on.\\
  \note{If you have chosen to install to an SD card and it is inserted at power
    on, Rockbox will boot from that card and use it as the primary drive for
    storing settings, etc. If there is no SD card inserted, Rockbox will boot
    from the internal memory, and it will not be possible to save settings.}
}

\opt{ipod}{
  \note{
    If you have loaded music onto your \dap{} using Itunes,
    you will not be able to see your music properly in the \setting{File Browser}.
    This is because Itunes changes your files' names and hides them in
    directories in the \fname{Ipod\_Control} directory. Files placed on your
    \dap{} using Itunes can be viewed by initialising and using Rockbox's database.
    See \reference{ref:database} for more information.
  }
}

\opt{iaudiom3}{
  \fixme{Add a note about the charging trick and place it here?}
}

\section{Updating Rockbox}
Rockbox can be easily updated with Rockbox Utility.
You can also update Rockbox manually -- download a Rockbox build
as detailed above, and unzip the build to the root directory
of your \dap{} as in the manual installation stage. If your unzip
program asks you whether to overwrite files, choose the ``Yes to all'' option.
The new build will be installed over your current build.\\

\opt{gigabeats}{
  \note{When your \dap{} is in the Rockbox USB or bootloader
  USB mode, you will see two visible partitions, the 150~MB firmware
  partition (containing at least a file called \fname{nk.bin}) and
  the main data partition. Rockbox \emph{must} be installed onto the main
  data partition.\\}
}

The bootloader only changes rarely, and should not normally
need to be updated.\\

\note{If you use Rockbox Utility be aware that it cannot detect manually
        installed components.}

\section{Uninstalling Rockbox}\index{Installation!uninstall}

\nopt{gigabeatf,iaudiom3,iaudiom5,iaudiox5,mrobe100,gigabeats,fuzeplus}{
  \note{The Rockbox bootloader allows you to choose between Rockbox and
  the original firmware. (See \reference{ref:Dualboot} for more information.)}
}

\subsection{Automatic Uninstallation}
\opt{gigabeats}{\note{Rockbox can only be uninstalled manually for now.}}

You can uninstall Rockbox automatically by using Rockbox Utility. If you
installed Rockbox manually you can still use Rockbox Utility for uninstallation
but will not be able to do this selectively.

\opt{iriverh100,iriverh300,fuzeplus}{\note{Rockbox Utility cannot uninstall the bootloader due to
the fact that it requires a flashing procedure. To uninstall the bootloader
completely follow the manual uninstallation instructions below.}}

\subsection{Manual Uninstallation}

\opt{iriverh10,iriverh10_5gb,mrobe100,vibe500,samsungyh}{
  If you would like to go back to using the original \playerman{} software,
  connect the \dap{} to your computer, and delete the
  \originalfirmwarefilename{} file and rename
  \fname{OF.mi4} to \originalfirmwarefilename{}
  in the \fname{System} directory on your \playertype{}.
  \nopt{mrobe100}{As in the installation,
  it may be necessary to first put your device into UMS mode.
  }
}

\opt{e200}{
  If you would like to go back to using the original \playerman{} software,
  connect the \dap{} to your computer, and follow the instructions to install
  the bootloader, but when prompted by sansapatcher, enter \texttt{u} for uninstall,
  instead of \texttt{i} for install. As in the installation, it may be necessary to
  first put your \dap{} into MSC mode.
}

\optv{ipod}{
  To uninstall Rockbox and go back to using just the original Ipod software, connect
  the \dap{} to your computer and follow the instructions to install
  the bootloader but, when prompted by ipodpatcher, enter \texttt{u} for uninstall
  instead of \texttt{i} for install.
}

\opt{iaudiom3,iaudiom5,iaudiox5}{
  If you would like to go back to using the original \playerman{} software,
  connect the \dap{} to your computer, download the original \playername{}
  firmware from the \playerman{} website, and copy it to the \fname{FIRMWARE}
  directory on your \playername{}. Turn off the \dap{}, remove the USB cable
  and insert the charger. The original firmware will automatically be flashed.
}

\opt{iriverh100,iriverh300}{
    If you want to remove the Rockbox bootloader, simply flash an unpatched
    \playerman{} firmware. Be aware that doing so will also remove the bootloader
    USB mode. As that mode can come in quite handy (especially if you experience
   disk errors) it is recommended to keep the bootloader. It also
    gives you the possibility of trying Rockbox anytime later by simply
    installing the distribution files.
    \opt{iriverh100}{
      The Rockbox bootloader will automatically start the original firmware if
      the \fname{.rockbox} directory has been deleted.
    }
    \opt{iriverh300}{
      Although if you retain the Rockbox bootloader, you will need to hold the
      \ButtonRec{} button each time you want to start the original firmware.
    }
}

    \opt{sansaAMS,fuzeplus}{
      Copy an unmodified original firmware to your player, and then reboot into
      the Sandisk firmware. See \reference{ref:Dualboot} for more information.
}

\opt{mpiohd200}{
    If you want to remove the Rockbox bootloader, simply flash an unpatched
    \playerman{} firmware.
}

\opt{fiiom3k,shanlingq1,erosqnative}{
    If you want to remove the Rockbox bootloader, copy an original firmware
    update to your microSD card and run the \playerman{} update by 
    \opt{erosqnative}{running it from the Original Firmware's System Settings menu.}
    \nopt{erosqnative}{holding \ActionBootOFRecovery{} while powering on the \dap{}.}

    Alternatively, if you took a backup of the original \playerman{} bootloader
    you can copy the backup file, \bootbackupfilename{}, to your SD card and
    select \emph{Restore bootloader} from the recovery menu. The recovery menu
    can be accessed by holding \ActionBootRecoveryMenu{} when powering on. If
    your \dap{} won't boot, use \fname{jztool} to load the bootloader over USB
    and enter the recovery menu -- see \reference{ref:jztool_load_bootloader}.
}

\nopt{gigabeats}{
  If you wish to clean up your disk, you may also wish to delete the
  \fname{.rockbox} directory and its contents.
  \nopt{iaudiom3,iaudiom5,iaudiox5,fiiom3k,shanlingq1}{
    Turn the \playerman{} off. Turn the \dap{} back on and the original
    \playerman{} software will load.}
}

\opt{gigabeats}{
  If you wish to clean up your disk by deleting the
  \fname{.rockbox} directory and its contents, this must be done
  before uninstalling the bootloader in the next step.

  Before installation you should have downloaded a copy of the \playerman{}
  firmware from
  \url{http://www.tacp.toshiba.com/tacpassets-images/firmware/MESV12US.zip}.
  \begin{itemize}
  \item Extract \fname{MES12US.iso} from the \fname{.zip} downloaded above.
  \item There are two files within \fname{MES12US.iso} called
  \fname{Autorun.inf} and \fname{gbs\_update\_1\_2\_us.exe}.  Extract them with
  your favourite unzipping utility e.g. 7zip.
  \item Connect your \dap{} to your computer.
  \item Extract \fname{nk.bin} from within
  \fname{gbs\_update\_1\_2\_us.exe} using e.g. 7zip and copy it to the 150~MB
  firmware partition of your \dap{}.
  \item Safely eject / unmount the USB drive, unplug the cable and restart.
  \end{itemize}

  \note{From Windows, you can also run \fname{gbs\_update\_1\_2\_us.exe}
  directly to restore your \dap{}. This will format your \dap{},
  removing all files.}
}

\section{Troubleshooting}
\begin{description}
\opt{sansa,ipod}{
  \item[Bootloader install problems]
  If you have trouble installing the bootloader,
  please ensure that you are either logged in as an administrator (Windows), or
  you have root rights (Linux)}

\opt{fuzeplus}{
  \item[Immediately loading original firmware.]
  If the original firmware is immediately loaded without going into updating
  the firmware, then the Rockbox bootloader has not been correctly installed.
  The original firmware will only perform the update if the filename is
  correct, including case. Make sure that the patched Sansa firmware is called
  \fname{firmware.sb} and present in the root directory of your player.
}

\opt{iriverh100,iriverh300}{
  \item[Immediately loading original firmware.]
  If the original firmware is immediately
  loaded without the Rockbox bootloader appearing first, then the Rockbox bootloader
  has not been correctly installed. The original firmware update will only perform
  the update if the filename is correct, including case. Make sure that the patched
  Iriver firmware is called \fname{.hex}.}

\nopt{iriverh100,iriverh300,mpiohd200}{\item[``File Not Found'']}
\opt{iriverh100,iriverh300,mpiohd200}{\item[``-1 error'']}
  If you receive a
  \nopt{iriverh100,iriverh300,mpiohd200}{``File Not Found''}
  \opt{iriverh100,iriverh300,mpiohd200}{``-1 error''}
  from the bootloader, then the bootloader cannot find the Rockbox firmware.
  This is usually a result of not extracting the contents of the \fname{.zip}
  file to the proper location, and should not happen when Rockbox has been
  installed with Rockbox Utility.

  To fix this, either install Rockbox with the Rockbox Utility which will take care
  of this for you, or recheck the Manual Install section to see where the files
  need to be located.
\end{description}

\optv{gigabeats}{
If this does not fix the problem, there are two additional procedures that you
can try to solve this:

\begin{itemize}
\item Formatting the storage partition. It is possible that using the
mkdosfs utility from Linux to format the data partition from your PC
before installing will resolve this problem. The appropriate format command is:
\begin{code}
    mkdosfs -f 2 -F 32 -S 512 -s 64 -v -n TFAT /path/to/partition/device
\end{code}
\warn{This will remove all your files.}

\item Copying a \fname{tar}. If you have a Rockbox build environment
then you can try generating \fname{rockbox.tar} instead of
\fname{rockbox.zip} as follows:
\begin{code}
    make tar
\end{code}
and copying it to the data partition. During the next boot, the bootloader
will extract it.
\end{itemize}
}


% $Id$ %
\opt{hotkey}{
    \section{\label{ref:Hotkeys}Hotkeys}
    Hotkeys are shortcut keys for use in the \nopt{touchscreen}{\setting{File Browser}
    and }\setting{WPS} screen.  To use one, press 
    \nopt{touchscreen}{\ActionTreeHotkey{} within the \setting{File Browser} or}
    \ActionWpsHotkey{} within the \setting{WPS}
    screen.\nopt{touchscreen}{ The assigned function will launch with reference
    to the current file or directory, if applicable.  Each screen has its own
    assignment.} If there is no assignment for a given screen,
    the hotkey is ignored.
    
    The default assignment for the \nopt{touchscreen}{File Browser hotkey is
    \setting{Off}, while the default for the }WPS hotkey is
    \setting{View Playlist}.
    
    The hotkey assignments are changed in the Hotkey menu (see
    \reference{ref:HotkeySettings}) under \setting{General Settings}.
}


